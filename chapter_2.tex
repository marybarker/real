\chapter{Measure on $[0, 1]$}

\begin{defn}\label{d:outermeasure}%19
	For any subset $E$ of $[0,1]$, the \textbf{outer measure} 
	of $E$ is 
	\begin{equation*}
		m^\ast(E) = \inf\left\{\sum\limits_{j=1}^\infty \ell(G_j) 
		: E \subset \bigcup\limits_{j=1}^\infty G_j\right\}
	\end{equation*}
	where each $G_j = (a_j,b_j)$ is an open interval in $\R$ and 
	$\ell(G_j) = b_j - a_j$. (Note: this includes finite sums by 
	setting all $G_j = \emptyset$ from some point on. Also note the $G_j$ 
	need not be contained in $[0,1]$, e.g. in considering $m^\ast(\{1\})$). 
\end{defn}

\begin{rmk}~ %20
	\begin{enumerate}[(a)]
		\item We could use coverings by closed intervals and get the same 
		result. It is clear that we could get numbers not larger than we 
		have. (Replace $G_j = (a_j, b_j)$ by $F_j = [a_j, b_j]$.) To 
		see not smaller, replace closed $F_j$ by open $G_j$ where 
		$\ell(G_j) = \ell(F_j) + \epsilon / 2^j$. 
		\item There is also something called inner measure. But it is not 
		needed, so we won't use it. 
	\end{enumerate}
\end{rmk}

\begin{pblm}%21
	For any subset $E$ of $[0,1]$, 
	\begin{enumerate}[(a)]
		\item $0 \le m^\ast(E) \le 1$.
		\item $E \subset F \Rightarrow m^\ast(E) \le m^\ast(F)$. 
		\item $m^\ast(\emptyset) = m^\ast(\text{one point})=0$.
	\end{enumerate}
\begin{proof}
~
\begin{enumerate}[(a)]
	\item For any $\epsilon > 0$, consider the open interval $(-\frac{\epsilon}{2}, 1 + \frac{\epsilon}{2})$ 
	that has length $1 + \epsilon$. This is an open cover of $[0,1]$, and so it is also an open 
	cover of $E \subset [0,1]$. Now, since $\epsilon$ was arbitrarily chosen, $m^\ast(E) \le 1$. 

	Also, since $m^\ast(E)$ is the infimum of a set of non-negative numbers, $m^\ast(E) \ge 0$. 

	So $0\le m^\ast(E)\le 1$ for all $E \subset [0,1]$. 

	\item Let $H=\left\{\sum\limits_{j=1}^\infty\ell(G_j):F\subset\bigcup\limits_{j=1}^\infty G_j\right\}$  
	and let $G=\left\{\sum\limits_{j=1}^\infty\ell(G_j):E\subset\bigcup\limits_{j=1}^\infty G_j\right\}$. 
	Since $E\subset F$, we know that $H\subset G$. Therefore $\inf(G)\le\inf(H)$. 

	\item For any $x \in [0,1]$, consider $(x - \epsilon / 2, x + \epsilon / 2)$ for any 
	$\epsilon > 0$. This is an open cover of $\{x\}$ with length $\epsilon$. Using the 
	fact that $\epsilon$ was arbitrarily chosen, the infimum of sums of lengths of intervals 
	must be less than or equal to $\epsilon$, which implies that $m^\ast(\{x\}) \le 0$. 

	Now, since $\emptyset \subset \{x\}$, we can use part (b), which gives 
	$m^\ast(\emptyset) \le m^\ast(\{x\}) \le 0$. 
\end{enumerate}
\end{proof}
\end{pblm}

\begin{pblm}%22
	If $E$ is a countable set, then $m^\ast(E) = 0$. In particular, 
	$m^\ast(\Q_0) = 0$ where $\Q_0 = \Q \cap [0,1]$ 
	is the set of all rational numbers in $[0,1]$.

 	(Given $\epsilon > 0$ cover $E = \{x_k\}_{k=1}^\infty$ with 
	$\bigcup\limits_{j=1}^\infty G_j$ where $\ell(G_j) = \epsilon / 2^j$.)
\begin{proof}
	Suppose $E$ is countable. Then we can write $E = \{x_i\}_{i=1}^\infty$. Then 
	for any $\epsilon > 0$, define $G_i = (x_i - \frac{\epsilon}{2^{i+1}}, x_i + \frac{\epsilon}{2^{i+1}})$. 
	Then $\ell(G_i) = 2\frac{\epsilon}{2^{i+1}}=\frac{\epsilon}{2^i}$. 
	Therefore $\bigcup\limits_{i=1}^\infty G_i$ is an open cover of $E$, and 
	\begin{equation*}
		\sum\limits_{i=1}^\infty\ell(G_i) = \sum\limits_{i=1}^\infty\frac{\epsilon}{2^i} = \epsilon. 
	\end{equation*}
	Therefore 
	\begin{equation*}
		m^\ast(E)\le\inf\left\{\sum\limits_{i=1}^\infty\ell(G_i):E\subset\bigcup\limits_{i=1}^\infty G_i\right\} = 0. 
	\end{equation*}
	and so $m^\ast(E) = 0$. 
\end{proof}
\end{pblm}

\begin{rmk}%23
	This is the first evidence that covering $E$ with countably infinite 
	unions rather than finite unions makes a big difference. This is, in 
	fact exactly the innovation that will allow us to deal much more 
	efficiently with infinite processes and with irregular sets. 
\end{rmk}

\begin{pblm}%24
	If $F = [a,b]$ is a closed interval in $[0,1]$, then $m^\ast(F)=b-a$. 
	($m^\ast(F) \le b - a$ is clear from $G = (a - \epsilon, b + \epsilon)$. 
	$F$ is compact, so any cover has a finite subcover $\bigcup\limits_{j=1}^n
	G_j$. May assume that no element of this subcover can be omitted (Explain 
	why.) Order them by their left endpoints and show $\sum\limits_{j=1}^n 
	\ell(G_j) > b - a.$)
\begin{proof}
	First we will show that $m^\ast(F) \le (b - a)$ and then that $m^\ast(F) \ge (b-a)$. 

	Now, $m^\ast(F) = \inf\left\{\sum\limits_{j=1}^\infty\ell(G_j):F\subset\bigcup\limits_{j=1}^\infty G_j\right\}$, and 
	so if we consider an arbitrary open cover $\{G_j\}_{j=1}^\infty$ of $F$, 
	$m^\ast(F) \le \sum\limits_{j=1}^\infty\ell(G_j)$. 
	In particular, Let us consider the open cover $(a - \epsilon / 2, b + \epsilon / 2)$ for 
	any $\epsilon > 0$. This has length $ (b - a) + \epsilon$. Thus 
	$m^\ast(F) \le (b - a) + \epsilon$ for all $\epsilon > 0$, and so 
	\begin{equation*}
		m^\ast(F) \le b - a
	\end{equation*}

	Now, let $\{G_j\}_{j = 1}^n$ be any finite subcover of $F$. Then if there are 
	$G_k = (a_k, b_k), G_l = (a_l, b_l) \in [0,1]$ such that $G_k \subset G_l$, 
	then we can throw out $G_k$. 

	Thus we can remove all extraneous $G_k$, and so if we reorder the elements 
	of $\{G_j\}_{j=1}^n$ in increasing order such that $a_0 < a $ and $b < b_n$. 
	With the increasing order for all $j$, $a_j < a_{j+1}$ and 
	$a_{j+1} < b_j < b_{j+1}$, since otherwise we would have 
	$(a_j,b_j) \subset (a_{j+1},b_{j+1})$. Then
	\begin{equation*}
		\sum\limits_{j=1}^n\ell(G_j) = \sum\limits_{j=1}^n (b_j-a_j)
	\end{equation*}
	This can be re-written as 
	\begin{equation*}
		\sum\limits_{j=1}^n(b_j-a_j) = (b_n - a_1) + (b_{n-1} - a_n) + (b_{n-2} - a_{n-1}) + \cdots
	\end{equation*}
	Now, since for all $j$, $b_j > a_{j+1}$, this is 
	\begin{equation*}
		b_n - a_1 + \epsilon
	\end{equation*}
	for some $\epsilon > 0$. Now, since $b = b_n + \epsilon_b$ and 
	$a_1 = a - \epsilon_b$ for some $\epsilon_a, \epsilon_b  > 0$, 
	we have 
	\begin{equation*}
		b_n + a_1 + \epsilon \ge b_n + a_1 = b - a + (\epsilon_b + \epsilon_a) \ge b - a. 
	\end{equation*}
	Continuing this with $\sum\limits_{j=1}^n\ell(G_j)$, we end with 
	\begin{equation*}
		\sum\limits_{j=1}^n \ell(G_j) \ge b - a
	\end{equation*}

	Since this open cover was arbitrarily chosen, $(b - a)$ is a lower bound, which means that since $m^\ast(F)$ is 
	the infimum, 
	\begin{equation*}
		m^\ast(F) = \inf\left\{\sum\limits_{j=1}^\infty\ell(G_j):F\subset\sum\limits_{j=1}^\infty G_j\right\} \ge (b - a)
	\end{equation*}
\end{proof}
\end{pblm}

\begin{pblm}%25
	If $E,F \subset [0,1]$, $m^\ast(E\cup F) \le m^\ast(E) + m^\ast(F)$. 
\begin{proof}
	Since the outer measure of any set $m^\ast$ is an infimum, it is a cluster 
	point. Therefore for any $\epsilon > 0$, there are open covers $\{G_j\}_{j=1}^\infty$ 
	and $\{H_j\}_{j=1}^\infty$ such that 
	\begin{equation*}
		\sum\limits_{j=1}^\infty\ell(G_j) = m^\ast(E) + \frac{\epsilon}{2} ~~~\text{ and }~~~ 
		\sum\limits_{j=1}^\infty\ell(H_j) = m^\ast(F) + \frac{\epsilon}{2}
	\end{equation*}
	Now, since $E \subset \bigcup\limits_{j=1}^\infty G_j$ and $F \subset \bigcup\limits_{j=1}^\infty H_j$, 
	$ E \cup F \subset \left(\bigcup\limits_{j=1}^\infty G_j\right) \cup \left(\bigcup\limits_{j=1}^\infty H_j\right)$ 
	and so 
	\begin{equation*}
		\begin{array}{rcccl}
			m^\ast(E \cup F) & \le & \sum\limits_{j=1}^\infty\ell(G_j) + \sum\limits_{j=1}^\infty\ell(H_j) 
					& = & m^\ast(E) + \frac{\epsilon}{2} + m^\ast(F) + \frac{\epsilon}{2}\\
					& & & = & m^\ast(E) + m^\ast(F) + \epsilon
		\end{array}
	\end{equation*}
	Thus $m^\ast(E\cup F) \le m^\ast(E) + m^\ast(F)$. 
\end{proof}
\end{pblm}

\begin{pblm}%26
	For any subsets $\{E_j\}_{j=1}^\infty$ of $[0,1]$, 
	$m^\ast\left(\bigcup\limits_{j=1}^\infty E_j\right) \le 
	\sum\limits_{j=1}^\infty m^\ast (E_j).$
	We refer to this property of outer measure as 
	\textbf{countable subadditivity}.
\begin{proof}
	Note that for any set of open covers of all of the $E_j$ is an open cover of $\cup E_j$. That is, for 
	$\left\{ \{I_{j_k}\}_{k=1}^\infty: E_j \subset \bigcup\limits_{k=1}^\infty I_{j_k}\right\}$, 
	$\bigcup\limits_{j=1}^\infty E_j \subset \bigcup\limits_{j=1}^\infty\bigcup\limits_{k=1}^\infty I_{j_k}.$ 

	For any $\{E_j\}_{j=1}^\infty$, consider any open covers $\{I_{j_k}\}_{k=1}^\infty$ 
	of each $E_j$. Then since each $m^\ast(E_j)$ is a cluster point, then for any $\epsilon > 0$ 
	there is $\{I_{j_k}\}_{k=1}^\infty$ such that 
	\begin{equation*}
		\sum\limits_{k=1}^\infty \ell(I_{j_k}) \le m^\ast(E_j) + \frac{\epsilon}{2^j}
	\end{equation*}
	And so for any $\epsilon > 0$, we can choose open covers of each of the $E_j$ such that 
	\begin{equation*}
		m^\ast\left(\bigcup\limits_{j=1}^\infty E_j\right) \le 
		\sum\limits_{j=1}^\infty\sum\limits_{k=1}^\infty \ell(I_{j_k}) \le 
		\sum\limits_{j=1}^\infty \left(m^\ast(E_j)+ \frac{\epsilon}{2^j}\right) = 
		\sum\limits_{j=1}^\infty m^\ast(E_j) + \epsilon
	\end{equation*}
	Therefore by (5), 
	\begin{equation*}
		m^\ast\left(\bigcup\limits_{j=1}^\infty E_j\right) \le \sum\limits_{j=1}^\infty m^\ast(E_j)
	\end{equation*}
\end{proof}
\end{pblm}

\begin{defn}\label{d:measurable}%27
	A subset $E$ of $[0,1]$ is \textbf{measurable} if for 
	each $A \subset [0,1]$, 
	\begin{equation*}
		m^\ast(A) = m^\ast(A\cap E) + m^\ast(A\cap E^c). 
	\end{equation*}
\end{defn}

\begin{notation}%28
	If $E$ is measurable, we write $m(E)$ for $m^\ast(E)$ and refer to 
	this number as the \textbf{Lebesgue measure} of $E$. 
\end{notation}

\begin{rmk}~ %29
	\begin{enumerate}[(a)]
		\item From the previous problem, to show that $E$ is measurable 
		one need only prove 
		$m^\ast(A) \ge m^\ast(A\cap E) + m^\ast(A\cap E^c)$ since the other 
		direction is automatic. 
		\item In particular, taking $A = [0,1], 1 = m^\ast(E)+m^\ast(E^c)$ 
		is a necessary condition for measurability. It turns out that this 
		special case of the definition is also sufficient. The definition 
		is the way it is (rather than this simpler version) just because 
		it is a little more convenient to use.
	\end{enumerate}
\end{rmk}

\begin{pblm}%30
{\color{gray!10}{thing}}
\begin{enumerate}[(a)]
	\item $\emptyset$ is measurable. $[0,1]$ is measurable.
	\item $m^\ast(E) = 0 \Rightarrow E$ is measurable. 
\end{enumerate}
\begin{proof}
{\color{gray!10}{thing}}
\begin{enumerate}[(a)]
	\item In $[0,1]$, $\emptyset^c = [0,1]$, and $[0,1]^c = \emptyset$. Therefore 
	for any $A \subset [0,1]$, 
	\begin{equation*}
		\begin{array}{rcccl}
		m^\ast(A) & = & m^\ast(A \cap [0,1]) & = & m^\ast(A \cap [0,1]) + 0\\
			 & = & m^\ast(A \cap [0,1]) + m^\ast(\emptyset) & = & m^\ast(A \cap [0,1]) + m^\ast(A \cap \emptyset)
		\end{array}
	\end{equation*}
	Thus $[0,1]$ is measurable. 
	Similarly, 
	\begin{equation*}
		\begin{array}{rcl}
		m^\ast(A) & = & 0 + m^\ast(A \cap [0,1])\\
			 & = & m^\ast(\emptyset)         + m^\ast(A \cap [0,1]) \\
			 & = & m^\ast(A \cap \emptyset)  + m^\ast(A \cap [0,1]) 
		\end{array}
	\end{equation*}
	Thus $\emptyset$ is measurable. 
	\item If $m^\ast(E) = 0$, then for any set $A$, $A \cap E \subset E$, and 
	since $A \cap E^c \subset A$, we have (by 21 (b)): 
	\begin{equation*}
		\begin{array}{rcl}
		m^\ast(A\cap E^c) & \le & m^\ast (A)\\
		0 + m^\ast(A\cap E^c) & \le & m^\ast (A)\\
		m^\ast(A \cap E) + m^\ast(A\cap E^c) & \le & m^\ast (A)
		\end{array}
	\end{equation*}
	Therefore, using the observation in 29(a), $E$ is measurable. 
\end{enumerate}
\end{proof}
\end{pblm}

\begin{rmk}%31
	In particular, every subset of a set of outer measure 0 is measurable. 
\end{rmk}

\begin{pblm}%32
	$E$ is measurable $\Leftrightarrow E^c$ is measurable. 
\begin{proof}
\begin{equation*}
	\begin{array}{rcl}
	E ~\text{ is measurable }~ & \Leftrightarrow & ~ \forall A \subset [0,1], m^\ast(A) = m^\ast(A \cap E) + m^\ast(A \cap E^c)\\
				& \Leftrightarrow & ~ \forall A \subset [0,1], m^\ast(A) = m^\ast(A \cap E^c) + m^\ast(A \cap E)\\
				& \Leftrightarrow & ~ \forall A \subset [0,1], m^\ast(A) = m^\ast(A \cap E^c) + m^\ast(A \cap (E^c)^c)\\
				& \Leftrightarrow & E^c ~\text{ is measurable}
	\end{array}
\end{equation*}
\end{proof}
\end{pblm}

\begin{pblm}%33
	A closed interval $[a,b]$ is measurable with measure $b-a$. 

	(Show $m^\ast(A) \ge m^\ast(A\cap[a,b])+m^\ast(A\cap[a,b]^c)-\epsilon$ 
	for every $\epsilon$. If $\bigcup\limits_{j=1}^\infty G_j$ covers $A$ 
	with $\bigcup\limits_{j=1}^\infty \ell(G_j)$ near $m^\ast(A)$, then the 
	$G_j\cap[a,b]$ are nearly an open cover of $A\cap[a,b]$ by open intervals 
	and similarly for $G_j\cap[a,b]^c$.)
\vspace{.25cm}

\begin{proof}
	Let $A \subset [0,1]$. Then 
	for any $\epsilon > 0$, choose an open cover $\{G_j\}_{j=1}^\infty$ of $A$ 
	such that $\sum\limits_{j=1}^\infty \ell(G_j) \le m^\ast(A) + \frac{\epsilon}{2}$. 

	Define $F_1 = \left\{G_k \cap (a, b) \right\}
		\cup (a-\frac{\epsilon}{8},a+\frac{\epsilon}{8})
		\cup (b-\frac{\epsilon}{8},b+\frac{\epsilon}{8})$, 
	$F_2 = \left\{G_k \cap [a, b]^c\right\}$, 

	Then $A \cap [a, b] \subset F_1$ and $A \cap [a, b]^c \subset F_2$. 
	(note that the elements of $F_2$ are either open intervals or else 
	sets of the form $(x, a) \cup (b, y)$ for some $x, y \in \R$.
	Therefore the notion of length is still well-defined on $F_2$ and $F_2$ 
	is still a collection of open intervals, and thus an open cover) 

	Since $\{O: O \subset F_1\}$ is an open cover of $A \cap [a, b]$ and 
	$\{O: O \subset F_2\}$ is an open cover of $A \cap [a, b]^c$, 
	Therefore, 
	$m^\ast(A \cap [a,b]) \le \sum\limits_{O \in F_1}\ell(O)$ and 
	$m^\ast(A \cap [a,b]^c) \le \sum\limits_{O \in F_2}\ell(O)$. 
	Also, since the only intersection of $F_1$ and $F_2$ is in the 
	$(a - \frac{\epsilon}{8}, a + \frac{\epsilon}{8})$, $(b - \frac{\epsilon}{8}, b + \frac{\epsilon}{8})$ 
	terms, then 
	\begin{equation*}
		\sum\limits_{O \in F_1 \cup F_2} \ell(O) 
		\le \sum\limits_{j = 1}^\infty \ell(G_j) + 
		\ell(a-\frac{\epsilon}{8},a+\frac{\epsilon}{8}) + \ell(b-\frac{\epsilon}{8},b+\frac{\epsilon}{8}) 
		= \sum\limits_{j=1}^\infty\ell(G_j)+\frac{\epsilon}{2}
	\end{equation*}
	Putting this together with the upper bounds for measures of $A \cap [a, b]$ and $A \cap [a, b]^c$, we get 
	\begin{equation*}
	m^\ast(A \cap [a,b]) + m^\ast(A \cap [a,b]^c) \le 
	\sum\limits_{O \in F_1 \cup F_2} \ell(O) = 
	\sum\limits_{k=1}^\infty \ell(G_k) + \frac{\epsilon}{2} 
	\end{equation*}
	And since $\sum\limits_{j=1}^\infty\ell(G_j) \le m^\ast(A) + \frac{\epsilon}{2}$, adding $\frac{\epsilon}{2}$ to 
	both sides of the inequality gives 
	$\sum\limits_{j=1}^\infty\ell(G_j) + \frac{\epsilon}{2} \le m^\ast(A) + \epsilon$, which we can substitute 
	into the previous equation to yield 
	\begin{equation*}
	m^\ast(A \cap [a,b]) + m^\ast(A \cap [a,b]^c) \le 
	\sum\limits_{k=1}^\infty \ell(G_k) + \frac{\epsilon}{2} 
	\le m^\ast(A) + \epsilon
	\end{equation*}

	and so $m^\ast(A \cap [a,b]) + m^\ast(A\cap[a,b]^c) \le m^\ast(A) + \epsilon$, which 
	can be re-written 
	\begin{equation*}
		m^\ast(A \cap [a,b]) + m^\ast(A\cap[a,b]^c) - \epsilon \le m^\ast(A)
	\end{equation*}
	For any $A \subset [0,1]$ for all $\epsilon > 0$. 
\end{proof}
\end{pblm}

\begin{pblm}%34
	If $E$ and $F$ are measurable, then so are $E \cap F$, $E \cap F^c$, 
	$E^c \cap F$, and $E^c \cap F^c$. 

	(To show $m^\ast(A) \ge m^\ast(A\cap E\cap F)+m^\ast(A\cap(E\cap F)^c)$ 
	use the measurability of $F$ with test set $A \cap E$. It is not 
	necessary to go back to the level of open covers.)
\begin{proof}
	Note first that For any sets $A, B, C$, 
	\begin{equation*}
		\{A \cap B \cap C^c \}
		\cup 
		\{A \cap B^c \cap C\}
		\cup 
		\{A \cap B^c \cap C^c\} = A \cap (B^c \cup C^c) = A \cap (B \cap C)^c
	\end{equation*}
	Now, since $E$ is measurable, for any $A \subset [0,1]$, 
	\begin{equation*}
	\begin{array}{rcl}
		m^\ast(A) & = & m^\ast(A \cap E) + m^\ast(A \cap E^c) \\
		& = & m^\ast(A \cap E \cap F) + m^\ast(A \cap E \cap F^c) \\
		& & + m^\ast(A \cap E^c \cap F) + m^\ast(A \cap E^c \cap F^c) \\
		& \ge & m^\ast(A \cap E \cap F) + m^\ast((A\cap E\cap F^c)
		{\scriptstyle \cup}(A\cap E^c\cap F){\scriptstyle \cup}(A\cap E^c\cap F^c)) \\
		& = & m^\ast(A \cap E \cap F) + m^\ast(A \cap (E \cap F)^c)  
	\end{array}
	\end{equation*}

	Now, since $E$ is measurable, $E^c$ is measurable, and similarly for $F^c$. Also, 
	Since the intersection of two arbitrary measurable sets is measurable (just shown), 
	then $E^c \cap F$, $E \cap F^c$ and $E^c \cap F^c$ are all measurable given that 
	$E$ and $F$ are. 
\end{proof}
\end{pblm}

\begin{pblm}%35
	If $E$ and $F$ are measurable, then so is $E \cup F$. If $E$ and $F$ are disjoint, 
	then $m(E \cup F) = m(E) + m(F)$. 
\begin{proof}
	if $E$ and $F$ are measurable, then so are $E^c$ and $F^c$, and so 
	$E^c \cap F^c$ is also. Therefore $E^c \cap F^c = (E \cup F)^c$ is also 
	measurable, so $E \cup F$ is also measurable. 

	Now, if $E$ and $F$ are measurable and disjoint, then taking the set $E \cup F$ 
	together with the measurability of $E$, (and given that $E \cup F$ is measurable also)
	\begin{equation*}
	\begin{array}{rcl}
		m(E \cup F) &=& m( (E \cup F) \cap E) + m( (E \cup F) \cap E^c)\\
			&=& m(E) + m(F)\\
	\end{array}
	\end{equation*}
\end{proof}
\end{pblm}

\begin{pblm}%36
	Any interval $\langle a,b\rangle$ with any choice of endpoints is 
	measurable with measure $b-a$. 
\begin{proof}
	From problem 33, $[a, b]$ is measurable with measure $b - a$. 
	On the other hand, if we consider the set $(a, b)$, 
	$m^\ast((a, b)) \le m^\ast([a,b]) = b - a$

	Now, from problem 21 (c), $m^\ast(\{a\}) = m^\ast(\{b\}) = 0$, and so 
	by 30(b), $\{a\}$ and $\{b\}$ are both measurable sets with measure 
	$0$. Therefore, since (by 35) $(a, b) \cap \{a\} = (a, b) \cap \{b\} = \emptyset$, and 
	\begin{equation*}
	\begin{array}{c}
		m^\ast((a, b)) = m^\ast( (a, b) \cup \{a\} \cup \{b\}) = m^\ast([a, b]) = b - a\\
			\\
		m^\ast(\left[\right.a, b\left.\right)) = m^\ast((a, b) \cup \{a\}) = m^\ast((a, b)) + m^\ast(\{a\})= m(a, b) = b - a\\
		\\
		m^\ast(\left(\right.a, b\left.\right]) = m^\ast((a, b) \cup \{b\}) = m^\ast((a, b)) + m^\ast(\{b\})= m(a, b) = b - a\\
	\end{array}
	\end{equation*}
\end{proof}
\end{pblm}

\begin{pblm}%37
	Any finite union or intersection of measurable sets is measurable. 
	If $\{E_k\}_{k=1}^n$ are measurable and disjoint, then 
	$m\left(\bigcup\limits_{k=1}^n E_k\right) = \sum\limits_{k=1}^nm(E_k)$. 
	(Induction).
\begin{proof}

\begin{itemize}
	\item Base case: proven in problem 35

	\item Induction step: Given that $m\left(\bigcup\limits_{k=1}^nE_k\right) = \sum\limits_{k=1}^nm(E_k)$, 
	and given that $E_{n+1}$ is measurable, then let $F = \bigcup\limits_{k=1}^nE_k$. 
	Since $E_{n+1}$ and $F$ are both measurable, then by 35, $\bigcup\limits_{k=1}^{n+1}E_k = E \cup F$ is also measurable. 
	If $E_{n+1}$ and $F$ are disjoint, then by 35, 
	\begin{equation*}
		m\left(\bigcup\limits_{k=1}^{n+1}E_k\right) = 
		m(E \cup F) = m(E) + m(F)
		= \sum\limits_{k=1}^{n+1}m(E_k)
	\end{equation*}
\end{itemize}
\end{proof}
\end{pblm}

\begin{pblm}%38 
	Let $E_1, E_2, ... , E_n$ be pairwise disjoint and measurable. Then 
	for any $A$, 
	\begin{equation*}
		m^\ast(A) = \sum\limits_{k=1}^nm^\ast(A\cap E_k)+m^\ast
		\left(A\cap\left(\bigcup\limits_{k=1}^nE_k\right)^c\right).
	\end{equation*}
\begin{proof}
	Base Case: Let $E_1, E_2$ be pairwise disjoint and measurable. Then for any $A$, 
	\begin{equation*}
	\begin{array}{rcccc}
		m^\ast(A) & = & m^\ast(A \cap (E_1 \cup E_2)) &+& m^\ast(A\cap(E_1\cup E_2)^c)\\
			& = & m^\ast\left((A \cap E_1)\cup (A \cap E_2)\right) & + & m^\ast(A \cap (E_1 \cup E_2)^c)\\
			& = & m^\ast(A \cap E_1) + m^\ast(A \cap E_2) & + & m^\ast(A\cap (E_1\cup E_2)^c)
	\end{array}
	\end{equation*}
	Now, assume that $E_1, E_2, ... , E_n$ are pairwise disjoint and measurable, and also 
	that $E_{n+1}$ is measurable and $E_i \cap E_{n+1} = \emptyset$ for all $i \in \{1, ... , n\}$. 
	Assume furthermore that for all $A$, 
	\begin{equation*}
		m^\ast(A) = \sum\limits_{k=1}^nm^\ast(A\cap E_k)+m^\ast
		\left(A\cap\left(\bigcup\limits_{k=1}^nE_k\right)^c\right).
	\end{equation*}
	Then $F = \bigcup\limits_{k=1}^nE_k$ is a measurable set that is disjoint from $E_{n+1}$, and therefore 
	for all $A$, 
	\begin{equation*}
	\begin{array}{rcl}
		m^\ast(A) &=& m^\ast(A \cap F) + m^\ast(A\cap E_{n+1}) + m^\ast(A \cap (F \cup E_{n+1})^c)\\
			&=&\sum\limits_{j=1}^nm^\ast(A\cap E_j) + m^\ast(A\cap E_{n+1}) + 
				m^\ast\left(A \cap \left(\bigcup\limits_{j=1}^nE_j \cup E_{n+1}\right)^c\right)\\
			&=&\sum\limits_{j=1}^{n+1}m^\ast(A\cap E_j) + m^\ast\left(A\cap\left(\bigcup\limits_{j=1}^{n+1}E_j\right)^c\right)
	\end{array}
	\end{equation*}
	Note that the expansion of $m^\ast(A\cap F)$ into a sum of measures from the first to second line 
	is due to the fact that all of the $E_j$ are disjoint, and so the $A \cap E_j$ are also, and so 
	37 states that the measure of the union is therefore equal to the sum of the measures. 
\end{proof}
\end{pblm}

\begin{pblm}%39
	Let $\{E_k\}_{k=1}^\infty$ be pairwise disjoint and measurable. Then 
	for any $A$, 
	\begin{equation*}
		m^\ast(A) = \sum\limits_{k=1}^nm^\ast(A\cap E_k)+m^\ast
		\left(A\cap\left(\sum\limits_{k=1}^nE_k\right)^c\right).
	\end{equation*}
	($\left\{\sum\limits_{k=1}^n m^\ast(A\cap E_k)\right\}_{n=1}^\infty$ 
	is a non-decreasing sequence that converges to 
	$\sum\limits_{k=1}^\infty m^\ast(A\cap E_k)$.  
	$\{s_n\}_{n=1}^\infty = 
	\left\{
		m^\ast \left(A\cap\left(\sum\limits_{k=1}^nE_k\right)^c\right)
	\right\}_{n=1}^\infty$ 
	is a non-increasing sequence all of whose terms are at least 
	$m^\ast \left(A\cap\left(\sum\limits_{k=1}^nE_k\right)^c\right).$ These 
	imply $m^\ast (A) \ge \sum\limits_{k=1}^\infty m^\ast(A\cap E_k) + 
	m^\ast \left(A\cap\left(\sum\limits_{k=1}^nE_k\right)^c\right).$ )
\begin{proof}
	First note that for every $n \in \Zp$, 
	\begin{equation*}
		m^\ast\left(\bigcup\limits_{j=1}^n(A\cap E_j)\right) = \sum\limits_{j=1}^n m^\ast(A \cap E_j)
	\end{equation*}
	Therefore, when we consider the sequence 
	$\left\{m^\ast\left(\bigcup\limits_{j=1}^n(A\cap E_j)\right)\right\}_{n=1}^\infty$, 
	this is equivalently written as $\left\{\sum\limits_{j=1}^nm^\ast(A\cap E_j)\right\}_{n=1}^\infty$. 

	Since $\left\{\sum\limits_{j=1}^nm^\ast(A\cap E_j)\right\}_{n=1}^\infty$ is a sequence of 
	non-decreasing partial sums each of which is bounded by $[0,1]$, we know the limit of this 
	sequence is bounded, and therefore (by 12)
	\begin{equation*}
		\lim\limits_{n\to\infty} m^\ast\left(\bigcup\limits_{j=1}^n(A\cap E_j)\right) = 
		\lim\limits_{n\to\infty} \sum\limits_{j=1}^nm^\ast(A\cap E_j) = \sum\limits_{j=1}^\infty m^\ast(A \cap E_j)
	\end{equation*}

	Now, since $\{S_n\}_{n=1}^\infty = 
	\left\{m^\ast\left(A \cap \left(\cup_{j=1}^n E_j\right)^c\right)\right\}_{n=1}^\infty$ 
	is a non-increasing sequence, all of whose terms are at least 
	$m^\ast\left(A\cap\left(\cup_{j=1}^\infty E_j\right)^c\right)$, then for every $n \in \Zp$, 
	\begin{equation*}
		\sum\limits_{j=1}^\infty m^\ast(A \cap E_j) + 
		m^\ast\left(A \cap \left(\bigcup\limits_{j=1}^n E_j\right)^c\right) 
		\ge 
		\sum\limits_{j=1}^\infty m^\ast(A \cap E_j) + 
		m^\ast\left(A \cap \left(\bigcup\limits_{j=1}^\infty E_j\right)^c\right)
	\end{equation*}
	and so 
	\begin{equation*}
	\begin{array}{l}
		\lim\limits_{n\to\infty} 
		\left\{
			m^\ast\left(\bigcup\limits_{j=1}^n(A\cap E_j)\right) +  
			m^\ast\left(A \cap \left(\bigcup\limits_{j=1}^n E_j\right)^c\right) 
		\right\} \\
			=  
			m^\ast\left(\bigcup\limits_{j=1}^\infty(A\cap E_j)\right) +  
			\lim\limits_{n\to\infty}
			m^\ast\left(A \cap \left(\bigcup\limits_{j=1}^n E_j\right)^c\right) \\ 
			=  
			\sum\limits_{j=1}^\infty m^\ast(A \cap E_j) + 
			\lim\limits_{n\to\infty}
			m^\ast\left(A \cap \left(\bigcup\limits_{j=1}^n E_j\right)^c\right) \\ 
			\ge  
			\sum\limits_{j=1}^\infty m^\ast(A \cap E_j) + 
			m^\ast\left(A \cap \left(\bigcup\limits_{j=1}^\infty E_j\right)^c\right)
	\end{array}
	\end{equation*}
	Therefore 
	\begin{equation*}
		m^\ast(A) \ge \sum\limits_{j=1}^\infty m^\ast(A \cap E_j) + m^\ast\left(A \cap \left(\bigcup\limits_{j=1}^\infty E_j\right)^c\right)
	\end{equation*}
	and so by the criterion in (25)
	\begin{equation*}
		m^\ast(A) = \sum\limits_{j=1}^\infty m^\ast(A \cap E_j) + m^\ast\left(A \cap \left(\bigcup\limits_{j=1}^\infty E_j\right)^c\right)
	\end{equation*}
\end{proof}
\end{pblm}

\begin{pblm}%40
	Let $\{E_k\}_{k=1}^\infty$ be measurable. Then 
	$\bigcup\limits_{k=1}^\infty E_k$ is measurable. If the $E_k$ are 
	pairwise disjoint, then 
	\begin{equation*}
		m\left(\bigcup\limits_{k=1}^\infty E_k\right)
		= \sum\limits_{k=1}^\infty m(E_k).
	\end{equation*}
	(Unions can be written as disjoint unions.) This property is called 
	\textbf{countable additivity}. 
\begin{proof}
	Let $\{E_j\}_{j=1}^\infty$ be measurable. Then by (4), we can define 
	\begin{equation*}
	\begin{array}{rcl}
		F_1 = E_1 & & F_k = E_k \setminus \bigcup\limits_{j=1}^{k-1}E_j \\
		\text{ where }~~ F\bigcup\limits_{j=1}^\infty F_j & = & \bigcup\limits_{j=1}^\infty E_j
	\end{array}
	\end{equation*}
	and since for all $n$, $F_n$ is a finite intersection of measurable sets, $F_n$ 
	is measurable. Therefore $\{F_j\}_{j=1}^\infty$ is a sequence of measurable sets 
	that are pairwise disjoint, so by (39), for any $A$, 
	\begin{equation*}
	\begin{array}{rcl}
		m^\ast(A) & = & \sum\limits_{k=1}^\infty m^\ast(A\cap F_k) + 
				m^\ast\left(A\cap\left(\bigcup\limits_{k=1}^\infty F_k\right)^c\right)\\
			 & = & m^\ast\left(\bigcup\limits_{k=1}^\infty (A\cap F_k)\right) + 
				m^\ast\left(A\cap\left(\bigcup\limits_{k=1}^\infty F_k\right)^c\right)\\
			 & = & m^\ast\left(A\cap\bigcup\limits_{k=1}^\infty F_k\right) + 
				m^\ast\left(A\cap\left(\bigcup\limits_{k=1}^\infty F_k\right)^c\right)\\
			 & = & m^\ast\left(A\cap\bigcup\limits_{k=1}^\infty E_k\right) + 
				m^\ast\left(A\cap\left(\bigcup\limits_{k=1}^\infty E_k\right)^c\right)\\
	\end{array}
	\end{equation*}
	So $\bigcup\limits_{k=1}^\infty E_k$ is measurable. 

	Now, assume that the $E_j$ are pairwise disjoint. Then letting 
	$F = \bigcup\limits_{k=1}^\infty E_k$, we can use the fact just proven that 
	$\bigcup\limits_{k=1}^\infty E_k$ is measurable together with the test set $F$ and by (39), 
	\begin{equation*}
	\begin{array}{rcl}
		m^\ast(F) &=& \sum\limits_{k=1}^\infty m^\ast(F\cap E_k) + 
				m^\ast\left(F\cap\left(\bigcup\limits_{k=1}^\infty E_k\right)^c\right)\\
		m^\ast(F) &=& \sum\limits_{k=1}^\infty m^\ast(F\cap E_k) + 
				m^\ast\left(F\cap(F)^c\right)\\
		m^\ast(F) &=& \sum\limits_{k=1}^\infty m^\ast(F\cap E_k) + m^\ast(\emptyset)\\
		m^\ast(F) &=& \sum\limits_{k=1}^\infty m^\ast(F\cap E_k) \\
		m^\ast\left(\bigcup\limits_{k=1}^\infty E_k\right) &=& 
			\sum\limits_{k=1}^\infty m^\ast\left(\left[\bigcup\limits_{j=1}^\infty E_j\right] \cap E_k\right) \\
		m^\ast\left(\bigcup\limits_{k=1}^\infty E_k\right) &=& 
			\sum\limits_{k=1}^\infty m^\ast(E_k) 
	\end{array}
	\end{equation*}
\end{proof}
\end{pblm}

\begin{pblm}\label{p:osouseful}%41
	Let $\{E_k\}_{k=1}^\infty$ be measurable and nested: $E_1 \subset E_2 
	\subset ... $ If $\bigcap\limits_{k=1}^\infty E_k = \emptyset$, then 
	$\lim\limits_{k\to\infty} m(E_k) = 0$. 

	($E_1$ is a disjoint union $\bigcup\limits_{k=1}^\infty F_k$ where 
	$F_1 = E_1 \setminus E_2$, $F_2 = E_2 \setminus E_3$ etc. 
	$m(E_n) = \sum\limits_{k=n}^\infty m(F_k)$.)
\begin{proof}
	Since these sets are nested, 
	\begin{equation*}
		m(E_1)= m(\bigcup_{k=1}^\infty E_k). 
	\end{equation*}
	Since $\{E_k\}_{k=1}^\infty$ is countable, let $F_k = E_k\setminus E_{k+1}$. 
	Then by problem 40, 
	\begin{equation*}
	m(E_1)=m(\bigcup_{k=1}^\infty E_k)=m(\bigcup_{k=1}^\infty F_k) = \sum_{k=1}^\infty F_k
	\end{equation*}
	Now, let $\bigcap _{k=1}^\infty E_k =\emptyset$.
	\begin{equation*}
		E_1\setminus E_k = \bigcup_{j=1}^{k-1} F_j
	\end{equation*}
	and since the $F_k$ are disjoint, 
	\begin{equation*}
		\bigcup_{j=1}^k E_j\setminus\bigcap_{j=1}^{k}E_j = \bigcup_{j=1}^{k-1} F_j=\bigcup_{j=1}^{k} F_j\setminus F_k
	\end{equation*}
	After rearranging the terms, we get 
	\begin{equation*}
		\bigcap_{j=1}^{k}E_j = \bigcup_{j=1}^k E_j\setminus\left(\bigcup_{j=1}^{k} F_j\setminus F_k \right)
	\end{equation*}
	Since these are countable unions and intersections of measurable sets, we have 
	\begin{equation*}
		m\left(\bigcap_{j=1}^{k}E_j\right) = m\left(\bigcup_{j=1}^k E_j\setminus\left(\bigcup_{j=1}^{k} F_j\setminus F_k \right)\right)
	\end{equation*}
	Taking the limit as $k$ goes to infinity, we have 
	\begin{equation*}
	\begin{array}{rcl}
		m(\emptyset) &=& m(E_1) - m(E_1) + \lim\limits_{k\to\infty}m(F_k)\\
		0 &=& \lim\limits_{k\to\infty}m(F_k)  \\
		& = & \lim\limits_{k\to\infty}m(E_k\setminus E_{k+1})
	\end{array}
	\end{equation*}
	Since $\bigcap _{k=1}^\infty E_k =\emptyset, \lim\limits_{k\to\infty}m(E_k\setminus E_{k+1}) =0$ 
	can only be true if $\lim\limits_{k\to\infty}m(E_k) = 0$ 
	therefore if $\{E_k\}_{k=1}^\infty$ are measurable and nested, then 
	\begin{equation*}
		E_1 \supset E_2 \cdots \textnormal{ and }\bigcap _{k=1}^\infty E_k =\emptyset
	\end{equation*} 
	then $\lim\limits_{k\to \infty} m(E_k) = 0$
\end{proof}
\end{pblm}

\begin{defn}\label{d:dotplus}%42
	If $E \subset [0,1]$ and $r$ is a real number, the n $E \Dp r$ is the set of 
	$\left[\left.0,1\right)\right.$ consisting of all fractional parts of the set 
	$\{x+r:x\in E\}$. Geometrically, add $r$ to each element of $E$ and then translate 
	the elements by an integer so that they land back in $\left[\left.0,1\right)\right.$. 
	(This can mean ``breaking the set into two parts.'' For instance $[0,.4] \Dp 
	\frac{3}{4} = [0,.15] \cup [3/4, 1).$)
\end{defn}

\begin{pblm}%43
	If $E$ is measurable and $r$ is a real number, then $E \Dp r$ is 
	measurable and $m(E\Dp r) = m(E)$. \\ (I found it helpful (i) to 
	show that if $A$ is any subset of $[0,1]$, (not necessarily measurable) 
	and $r \in \R$ is such that $A + r \subset[0,1]$, then 
	$m^\ast(A) = m^\ast(A + r)$ and (ii) to remember that we have already 
	seen in a previous problem that if $A$ is any subset of $[0,1]$ and 
	$s \in (0,1)$ then $m^\ast(A) = m^\ast(a \cap [0,s]) + m^\ast(A\cap(s,1])$.)
\begin{proof}
\end{proof}
\end{pblm}

\begin{rmk}%44
	We can sum up what we have done so far like this. \\
	The set $\M$ of Lebesgue measurable subsets of $[0,1]$ has the following 
	properties
	\begin{enumerate}
		\item $\emptyset \in \M$
		\item Every interval is in $\M$ with $m\left(\langle a, b\rangle\right) = b - a$. 
		Moreover every open subset of $[0,1]$ is in $\M$. 
		\item $E \in \M \Rightarrow E^c \in \M$. 
		\item $E \in \M \Rightarrow E\Dp r \in \M$ and $m(E \Dp r) = m(E)$. 
		\item If $\{E_k\}_{k=1}^\infty$ are in $\M$, then $\bigcup\limits_{k=1}^\infty E_k \in 
		\M$. 
		\item If, in addition, the $E_k$ are pairwise disjoint, then $m\left(\bigcup\limits_{k=1}^\infty 
		E_k \right) = \sum\limits_{k=1}^\infty m(E_k)$
	\end{enumerate}
\end{rmk}

\begin{ex}%45
	ince the set $\Q_0$ of rational numbers in $[0,1]$ is countable, $\Q_0$ 
	is measurable and $m(\Q_0) = 0$. Then $\Q_0^c$-the set of irrational 
	numbers in $[0,1]$-is also measurable, and 
	$m(\Z_0^c) = m([0,1]) - m(\Q_0) = 1$. Thus the ``length'' of the 
	irrationals is 1, while the ``length'' of the rationals is 0. This is connected to the fact 
	that $\Q_0$ is countable while $\Q_0^c$ is uncountable, but it would 
	be wrong to assume that all uncountable subsets of $[0,1]$ have measure 1, or even 
	nonzero measure. See the next example. 
\end{ex}

\begin{ex}%46
	Recall that the Cantor set $C$ is the subset of $[0,1]$ that remains after removing 
	$\bigcup\limits_{k=1}^\infty I_k$ where $I_1 = \left(\frac{1}{3},\frac{2}{3}\right)$, 
	$I_2 = \left(\frac{1}{9},\frac{2}{9}\right) \cup \left(\frac{7}{9},\frac{8}{9}\right)$ 
	and in general $I_k$ is the union of the $2^k$ open middle thirds of the closed intervals 
	remaining after $I_{k-1}$ has been removed. Furthermore, $m(I_k) = \frac{1}{3}\left(\frac{2}{3}\right)^{k-1}$ 
	so that $m\left(\bigcup\limits_{k=1}^\infty I_k\right) = \frac{1}{3}\sum\limits_{k=0}^\infty 
	\left(\frac{2}{3}\right)^k = 1$ and $m(C) = 0$. 
\end{ex}

\begin{rmk}%47
	It is not immediately obvious from the construction above that $C$ is uncountable. 
	Some of you at least have seen an alternative characterization of $C$ as the set 
	of all numbers in $[0,1]$ with a ternary expansion containing only 0's and 2's. 
	Since this set can be put in 1-1 correspondence with the set of all binary expansions 
	of numbers in $[0,1]$, that is, with $[0,1]$, it is true that $C$ is uncountable. Thus 
	the Lebesgue measure of a set is not particularly connected to the cardinality of 
	the set. These are two rather different versions of the ``size'' of an infinite set. 
\end{rmk}

\begin{pblm}%48
	By altering the lengths of the removed intervals, construct a ``fat Cantor set''-a 
	Cantor set of positive measure. Ideally, show how to construct a Cantor set of 
	measure $\alpha$ for any $\alpha < 1$. \\
	(You can check that if you try to imitate the Cantor set construction by 
	removing a constant fraction of what remains at each stage, the total length of 
	removed intervals will be 1 no matter what the fraction removed is, so you will 
	have to be a little cleverer than that. Try to make choices so that your 
	calculations are reasonably easy.)

\begin{proof}
	For any $\alpha < 1$, let $\beta = 1 - \alpha$. We will construct the 
	complement $C_f^c$ of a ``fat Cantor set $C_f$'' that has measure $\beta$ 
	so that the measure of $C_f$ is $1 - \beta = \alpha$. 

	To construct this set, as with the Cantor set, we will use the intervals 
	that are discarded from the center of a given interval at each step. 

	For the first step, remove the interval of length $\frac{\beta}{2}$ from 
	the center of the interval $[0,1]$. Then at each successive step, remove the 
	intervals of length $\frac{\beta}{2^{2n}}$ from the center of each remaining 
	open interval. 

	\begin{center}
	\begin{tikzpicture}
	\draw (0,0) -- (10,0);
	\node at (0.0, 0) {$($};
	\node at (10., 0) {$)$};
	\node at ( 0, .5) {0};
	\node at (10, .5) {1};
	\draw [thick] (4,0) -- (6,0);
	\node at (4.0, 0) {$($};
	\node at (6.0, 0) {$)$};
	\node at (5,1) {$\frac{\beta}{2}$};
	\draw [thick] (1.5, 0) -- (2.5, 0);
	\node at (1.5, 0) {$($};
	\node at (2.5, 0) {$)$};
	\node at (2,1){$\frac{\beta}{2^2}$};
	\draw [thick] (7.5, 0) -- (8.5, 0);
	\node at (7.5, 0) {$($};
	\node at (8.5, 0) {$)$};
	\node at (8,1){$\frac{\beta}{2^2}$};
	\end{tikzpicture}
	\end{center}
	The intervals removed form a set that has a sum length of intervals of 
	\begin{equation*}
		\frac{\beta}{2} + 2\frac{\beta}{4^2} + \dots = \beta\sum\limits_{i=1}^\infty \frac{1}{2^i} = \beta
	\end{equation*}
	which means that the complement of this set has measure $1 - \beta = \alpha$. 
\end{proof}
\end{pblm}

\begin{rmk}%49
	Properties (1),(3),(5) of \#44 are really structural properties of the collection 
	$\M$ of measurable subsets of $[0,1]$. How interesting they are depends 
	on whether $\M$ is just the set of all subsets of $[0,1]$ (sometimes 
	called the \textbf{power set} $\Sp([0,1])$) or some proper collection of 
	$\Sp([0,1])$. We ``construct'' some non-measurable sets in order to see 
	that $\M \neq \Sp([0,1])$. This makes the structural properties 
	much more interesting. 

	Given $x \in [0,1]$, let $A_x = \{y \in [0,1] : x - y \in \Q\}$. 
	Each $A_x$ is a countable set. Clearly $y \in A_x \Leftrightarrow 
	x \in A_y$. If $x \in A_y$ and $y \in A_z$, then $x - z = (x + y) 
	+ (y - z)$ is rational so $z \in A_x$. It follows that the relation 
	$x \sim y$ if $y \in A_x$ is reflexive, symmetric and transitive, 
	that is, it is an \textbf{equivalence relation}. Each distinct set 
	$A_x$ is an \textbf{equivalence class}. As you have probably 
	verified somewhere else, distinct equivalence classes are disjoint, 
	that is, if $x \neq y$ then either $A_x = A_y$ or $A_x$ and $A_y$ 
	are disjoint. (Easy if you haven't done it before.)

	Now choose one element from each distinct equivalence class, and 
	let $E$ be the set of all such elements. $E$ is an uncountable set; 
	otherwise $[0,1]$ would be a countable union of countable sets, and 
	so countable. See the remark below about hte choice process. 

	Let $\{q_k\}_{k=1}^\infty$ be an enumeration of the set $\Q_0$ of 
	rationals in $[0,1]$, I claim the sets $\{E \Dp q_k : k = 1, 2, 3\}$ 
	are disjoint and that their union is $[0,1]$. First note that for 
	any $x \in [0,1]$, $x \in E \Dp q_k$ if for some $e \in E$, either 
	$x = e + q_k$ (if $e + q_k < 1$) or $x = e + q_k - 1$ (in case 
	$e + q_k \ge 1$). Now $x$ is in the same equivalence class $A_x$ as 
	some element $e$ of $E$. Thus $|x - e| = q_k$ for some $k$ and we 
	see that either $x \in E \Dp q_k$ (if $x \ge e$) or $x \in E \Dp 
	(1 - q_k)$ (if $x < e$). Next we see that the $\{E \Dp q_k\}$ are 
	pairwise disjoint. If $x = e + q_k = e^t + q_j$, then $e - e^t = 
	q_j - q_k$, that is, $e$ and $e^t$ are in the same equivalence 
	class. This is a contradiciton. 

	To summarize, $[0,1] = \bigcup\limits_{k=1}^\infty (E \Dp q_k)$ as 
	a disjoint union. Further, by property 4 of \#44, either all these 
	sets are measurable with the same measure, or all are non - 
	measurable. But the first alternative is impossible by property 6 
	of \#44. Thus we have a countably infinite collection of non-measurable 
	sets. 
\end{rmk}

\begin{rmk}%50
	The construction of $E$ in the preceding example used the property 
	that I can make a set by choosing one element out of a collection of 
	non-empty sets. The assertion that one can do this is called the 
	\textbf{Axiom of Choice}. In the early twentieth century when 
	mathematicians and logicians were trying to construct all of 
	mathematics from a precisely defined set of axioms, and to show that 
	the mathematics so obtained would be consistent (no possibility that 
	legal arguments would produce a contradiction), the relationship 
	between the Axiom of Choice and other ``usual axioms'' was much studied, 
	because the Axiom of Choice has surprising consequences (e.g. that it 
	is possible to order the set of real numbers so that each subset of 
	reals has a least element in the ordering-just like the positive integers 
	with the usual ordering. This is called a Well-Ordering.) However in 
	1963 Paul Cohen completed the proof that AofC is independent of the other 
	axioms of set theory. This marked the very end of the interest that most 
	mathematicians have in the foundations of mathematics. (Most feel that it 
	is new ideas and connections that are of interest, and Cohen's result 
	showed that AofC, which is often very convenient, can't ``make things 
	worse'' (introduce contradictions where there were none before.)).
\end{rmk}

\begin{defn}\label{d:sigmaalgebra}%51
	A collection $\A$ of subsets of a non-empty set $X$ is a 
	$\sigma-$\textbf{algebra} of subsets if $\A$ has the properties 
	\begin{enumerate}[(i)]
		\item $\emptyset \in \A$, 
		\item $E \in \A \Leftrightarrow E^c \in \A$
		\item if $\{E_k\}_{k=1}^\infty$ are in $\A$, then 
			$\bigcup\limits_{k=1}^\infty E_k \in \A$
	\end{enumerate}
\end{defn}

\begin{rmk}%52
	Thus $\M$ is a proper $\sigma$-algebra of subsets of $\Sp([0,1])$, 
	the power set of $[0,1]$. Another $\sigma-$algebra of subsets of 
	$\Sp([0,1])$ is the collection $\B$ of all \textbf{Borel subsets} of 
	$[0,1]$. $\B$ is defined to be the smallest $\sigma-$algebra of subsets 
	of $\Sp([0,1])$ that contains the open sets. (Take the intersection of all 
	such $\sigma-$algebras.) Clearly $\B\subset\M$. It can be shown that $\B$ 
	and $\M$ are different by a cardinality argument-Since the Cantor set $C$ 
	can be put into 1-1 correspondence with $\Sp([0,1])$. All elements in 
	$\Sp(C)$ are measurable, since they are subsets of a set of measure 0, 
	thus $\M$ has the same cardinality as $\Sp([0,1])$. On the other hand, $\B$ 
	can also be generated in a countable fashion from the set of open intervals 
	with rational endpoints. Thus its cardinality is the same as that of 
	$[0,1]$ and so strictly less than that of $\Sp([0,1])$. 

	On the other hand, every measurable set is ``nearly'' an open set, and 
	also ``nearly'' a closed set. This is the content of the next problem. 
\end{rmk}

\pagebreak
\begin{pblm}%53
	The following properties are equivalent for a subset $E$ of $[0,1]$
	\begin{enumerate}[(i)]
		\item $E \in \M$
		\item for each $\epsilon > 0$ there is an open set $G$ such that 
		$E \subset G$ and $m^\ast(G\setminus E) < \epsilon$. 
		\item there is a $G_\delta$ set $H$ such that $E \subset H$ and 
		$m^\ast(H\setminus E) = 0$. \\
		(A $G_\delta$ set is a countable intersection of open sets. A 
		$G_\delta$ is a Borel set but not open in general, for instance 
		any closed interval is a $G_\delta$. (Proof?-but not a problem))
		\item for each $\epsilon > 0$ there is a closed set $F$ such that 
		$F \subset E$ and $m^\ast(E\setminus F) < \epsilon$. 
		\item there is an $F_\sigma$ set $K$ (a countable union of closed 
		sets, e.g. $\Q_0$) such that $K \subset E$ and 
		$m^\ast(E\setminus K) = 0$. 

		(We will split this into two parts, from two different people, 
		first (i) $\Rightarrow$ (ii) $\Rightarrow$ (iii) $\Rightarrow$ (i), 
		second (ii) $\Rightarrow$ (iv) $\Rightarrow$ (v) $\Rightarrow$ (i).)
	\end{enumerate}
\begin{proof}
\noindent $(i) \Rightarrow (ii)$: If $E$ is a measurable set, then for each 
	$\epsilon > 0$ there is an open cover $\{G_j\}_{j=1}^\infty$ of $E$ such that 
	\begin{equation*}
		m(E) < \sum\limits_{j=1}^\infty \ell(G_j) < m(E) + \epsilon
	\end{equation*}
	Now, let $G = \bigcup\limits_{i=1}^\infty G_j$. Then 
	$G$, as a union of open sets, is itself open. Also, since the $G_j$ were 
	chosen as an open cover of $E$,  $E \subset G$, and so (since $G\setminus E$ 
	and $E$ are mutually disjoint), 
	\begin{equation*}
		m^\ast(G \setminus E) + m^\ast(E) \le m^\ast(G) \le m^\ast(E) + \epsilon
	\end{equation*}
	which can be re-written 
	\begin{equation*}
		m^\ast(G \setminus E) + m^\ast(E) \le m^\ast(E) + \epsilon ~~ \Rightarrow ~~ m^\ast(G \setminus E) \le \epsilon
	\end{equation*}

\noindent $(ii) \Rightarrow (iii)$: 
	Let $H=\bigcup\limits_{k=1}^\infty H_i$ where $E\subset H_i,\,\forall i$ and (from part (ii)): 
	\begin{equation*}
		m^\ast(H_i\setminus E) = \frac{1}{i}
	\end{equation*}

	Then $E\subset H$ and $H = \bigcup_{k=1}^\infty H_i$ is a $G_\delta$ set such that 
	$ m^\ast(H\setminus E) = \lim\limits_{i\to\infty} \frac{1}{i} = 0. $

\noindent $(iii) \Rightarrow (i)$: 
	Given a $G_\delta$ set $H$, such that $E \subset H$, and $m^\ast(H\setminus E) = 0$, 
	Then $H$ is measurable since $\M$ is a $\sigma$-algebra. 
	\begin{equation*}
		E = (H\setminus E)^c \cap H.
	\end{equation*}
	Notice all the constituent sets are measurable, thus $E\in\M$.  

\noindent $(ii) \Rightarrow (iv)$: 
	Let $\epsilon > 0$. Then there is an open set $G$ such that 
	$E\subset G$ and $m^\ast(G\setminus E) < \epsilon$.

	 Then $E^c$ is also measurable 
	and so there is an open set $F^c$ such that $E^c\subset F^c$ and 
	$m^\ast(F^c\setminus E) \le \epsilon$. This implies $F\subset E$. 

	Then $m^\ast(F^c\setminus E^c)=m^\ast(F^c\cap E) = m^\ast(E\cap F^c) < \epsilon$.

\noindent $(iv) \Rightarrow (v)$: 
	Let $K\subset E$ such that $K=\bigcup_{k=1}^\infty F_i$, and each $F_i\in E$  
	$m^\ast(E\cap F_i^c) < \frac{\epsilon}{2^k}$. Then $E\setminus K \subset F_i$. 
	Furthermore, 
	\begin{equation*}
		m^\ast(E\setminus K) < m^\ast(E\setminus F_i) < \frac{\epsilon}{2^k} < \epsilon 
	\end{equation*}
	for all $1 \le k \le \infty$ 

\noindent $(v) \Rightarrow (i)$: 
	Let $K$ be an $F_\sigma$ set such that $K\subset E$ and $m^\ast(E\setminus K)= 0$. 
	Then $m^\ast(E\setminus K)$ is measurable. Also, since $K$ is an $F_\sigma$ set, 
	it is a countable union of closed sets and is therefore measurable. 
	Then $E = K\cup E\setminus K$ is the union of measurable sets. 
	Thus $E$ is measurable. 

\end{proof}
\end{pblm}


