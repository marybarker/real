\chapter{Preliminaries}
\begin{defn}\label{d:finite}%1
A set $A$ is \textbf{finite} if there is a 1-1 mapping of some set $\{1, 2, 3, ..., n\}$ 
of positive integers onto $A$. $A$ is \textbf{countably infinite} if there is a 1-1 
mapping of the set $\Zp$ of positive integers onto $A$. $A$ is \textbf{countable} 
if either finite or countably infinite. Otherwise $A$ is \textbf{uncountable}. 
\end{defn}

\begin{pblm}%2
	(deMorgan laws) If $\{E_\lambda: \lambda \in \Lambda\}$ is any indexed collection of 
	subsets of some set $E$, then 
	\begin{equation*}
		\left(\bigcup\limits_{\lambda \in \Lambda} E_\lambda\right)^c = 
		\bigcap\limits_{\lambda \in \Lambda} E_\lambda^c ~~~ \text{ and }~~~
		\left(\bigcap\limits_{\lambda \in \Lambda} E_\lambda\right)^c = 
		\bigcup\limits_{\lambda \in \Lambda} E_\lambda^c 
	\end{equation*}
	where $A^c$ denotes the complement of $A$ in $E$. 
\begin{proof}
	\begin{equation*}
	\left(\bigcup\limits_{\lambda \in \Lambda} E_\lambda\right)^c = 
	\{x: x \in E_\lambda \text{ for some } \lambda \in \Lambda\}^c = 
	\{x: x \in E_\lambda^c \forall \lambda \in \Lambda\} = 
	\bigcap\limits_{\lambda \in \Lambda} E_\lambda^c
	\end{equation*}
	\begin{equation*}
	\left(\bigcup\limits_{\lambda \in \Lambda} E_\lambda\right)^c = 
	\{x: x \in E_\lambda \text{ for some } \lambda \in \Lambda\}^c = 
	\{x: x \in E_\lambda^c \forall \lambda \in \Lambda\} = 
	\bigcap\limits_{\lambda \in \Lambda} E_\lambda^c
	\end{equation*}
\end{proof}
\end{pblm}

\begin{rmk}%3
	In the previous problem the index set $\Lambda$ need not be countable. One could 
	imagine indexing a collection of sets by the real numbers, for instance (e.g. 
	$E_x$ is the interval of length 1 centered at $x$.)
\end{rmk}

\pagebreak
\begin{pblm}%4
	Any countable union of sets of real numbers can be expressed as a disjoint 
	union: $E \cup F = E \cup (F \setminus E)$ or $\bigcup\limits_{k = 1}^\infty 
	E_k = \bigcup\limits_{k = 1}^\infty F_k$ where $F_1 = E_1$, 
	$F_k = E_k \setminus \bigcup \limits_{j = 1}^{k - 1} E_j$. Here $A \setminus B = 
	A \cap B^c$. 
\begin{proof}
	Let $E, F$ be sets of real numbers. 
	Then for any $x \in E \cup F$, if $x \in E$, then 
	$x \in E \subset \{E \cup (F \setminus E)\}$. Similarly, if $x \notin E$ but 
	$x \in F$, then $x \in (F \setminus E) \subset \{E \cup (F \setminus E)\}$. 
	Thus 
	\begin{equation*}
		E \cup F ~\subset~ E \cup (F\setminus E). 
	\end{equation*}

	On the other hand, for any $x \in E \cup (F \setminus E)$, if $x \in E$ then 
	$x \in E \cup F$. Also, if $x \in F \setminus E$ then $x \in F \subset (E \cup F)$. 
	Thus 
	\begin{equation*}
		E \cup (F \setminus E) \subset E \cup F. 
	\end{equation*}

	\begin{equation*}
		E \cup F ~ \substack{\subset \\ \supset}~ E \cup (F \setminus E) ~~~\implies~~~
		E \cup F = E \cup (F \setminus E).
	\end{equation*}

	For the more general case, take any $x \in \bigcup\limits_{j=1}^\infty F_k$, then there is some $F_j$ such that 
	$x \in F_j$. Since each $F_j \subset E_j$, then 
	$x \in E_j \subset \bigcup\limits_{j=1}^\infty E_j$. Therefore 
	$\bigcup\limits_{j=1}^\infty F_j\subset \bigcup\limits_{j=1}^\infty E_j$. 

	On the other hand, for any $x \in \bigcup\limits_{j=1}^\infty E_j$. Then there is 
	some $k$ such that $x \in E_k$. Let $a$ be the 
	smallest index such that $x \in E_a$. Then $x \in E_a$ and $\forall b < a$, $x 
	\notin E_b$. Therefore $x \in F_a$, and so $x \in \bigcup\limits_{j=1}^\infty F_j$. 
	Therefore, since we have shown subset containment in both directions, 

	\begin{equation*}
		\bigcup\limits_{j=1}^\infty E_j = \bigcup\limits_{j=1}^\infty F_j. 
	\end{equation*}
\end{proof}
\end{pblm}

\begin{pblm}%5
	If $a$ and $b$ are real numbers, then $a \le b$ iff for every $\epsilon > 0$, 
	$a \le b + \epsilon$. Similarly, $a \ge b$ iff for every $\epsilon > 0$, 
	$a \ge b - \epsilon$. 
\begin{proof}
	Let $a, b \in \R$ and $a \le b$. Then for every $\epsilon > 0$, 
	$a \le b < b + \epsilon$, and therefore $a \le b + \epsilon$. 
	Now, let $a, b \in \R$, and assume that for any 
	$\epsilon > 0$, $a \le b + \epsilon$. Then if $a > b$, there is some 
	$\epsilon_1 \ge 0$ such that $a = b + \epsilon_1$. But since $a \le b + \epsilon$ for 
	all $\epsilon > 0$, we have 
	$a \le b + \frac{\epsilon}{2} < b + \epsilon_1 = a $ 
	which is a contradiction. Therefore $a \le b$. 

	Now let $a \ge b$. This can be re-written as $b \le a$. By 
	the first part of the problem, $b \le a$ iff $\forall \epsilon > 0$, 
	$b \le a + \epsilon$. This is true iff $b - \epsilon \le a$. 
\end{proof}
\end{pblm}

\begin{rmk}%6
	This problem may seem trivial, but we will use it over and over again during the quarter. 
	It says that we can give a little something away, and then take it back. 
\end{rmk}

\begin{defn}\label{d:boundedset}%7
~
	\begin{enumerate}[(a)]
		\item A set $E$ of real numbers is \textbf{bounded above} if there 
		exists a real number $u$ such that for every $x \in E$, $x \le u$. 
		we call $u$ an \textbf{upper bound} for $E$. Similarly, $E$ is 
		\textbf{bounded below} if there exists a real number $w$ such that 
		for every $x \in E$, $w \le x$, and then $w$ is a \textbf{lower bound} 
		for $E$. $E$ is \textbf{bounded}, if bounded both above and below. 
		\item If $E$ is bounded above and nonempty, the \textbf{supremum} 
		(or least upper bound) of $E$, sup $E$, is the unique real number $s$ 
		such that (i) $s$ is an upper bound for $E$ and (ii) $s < u$ for any 
		other upper bound $u$ for $E$. Similarly, if $E$ is bounded below 
		and nonempty, the \textbf{infimum} (or greatest lower bound) of $E$, 
		inf $E$, is the unique real number $t$ such that (i) $t$ is a lower 
		bound for $E$ and (ii) $t > w$ for any other lower bound $w$ for $E$. 
		(Recall that the Least Upper Bound Axiom guarantees the existence of 
		the supremum and infimum.) 
		\item A real number $c$ is a \textbf{cluster point} of a set $E$ of 
		real numbers if for every $\epsilon > 0$ there is $y \in E$ such that 
		$0 < |c - y| < \epsilon$. 
	\end{enumerate}
\end{defn}

\begin{pblm}%8
	Let $v$ be a lower bound for a set $E$ of real numbers, where $v \notin E$. Then 
	$v = \inf E$ iff $v$ is a cluster point of $E$. 
\begin{proof}
	Let $v = \inf E$, $v \notin E$. Then for any $\epsilon > 0$, 
	there is $y \in E$ such that $y < v + \epsilon$, as otherwise $v + \epsilon$ would 
	be a lower bound of $E$ greater than $v = \inf E$. Now, for any $\epsilon > 0$, we 
	have $y \in E$ such that 
	\begin{equation*}
		v < y < v + \epsilon
	\end{equation*}
	Subtracting $v$ from each inequality, this can be re-written as 
	\begin{equation*}
		0 < y - v < \epsilon
	\end{equation*}
	and since $y > v$, this is 
	\begin{equation*}
		0 < |y - v| < \epsilon ~~~ \Leftrightarrow ~~~ 0 < |v - y| < \epsilon 
	\end{equation*}
	Thus $v$ is a cluster point of $E$. 

	Now, assume that $v$ is a cluster point of $E$, $v \notin E$, and $v$ is a L.B. for 
	$E$. 

	For any $w > v$, $w = v + \epsilon$ for some $\epsilon > 0$, there is $y \in E$ such that 
	$v < y < v + \epsilon$. Therefore $w$ is not a lower bound of $E$. Thus $v = \inf E$. 
\end{proof}
\end{pblm}

\pagebreak
\begin{defn}\label{d:boundedsequence}%9
	~
	\begin{enumerate}[(a)]
		\item A \textbf{sequence} of real numbers $\{a_k\}_{k=1}^\infty$ is 
		a function from the set of positive integers $\Zp$ into the 
		real numbers. (More generally, the domain of a sequence can be any set 
		of the form $\{k: k \ge k_0\}$ for some integer $k_0$.) The sequence is 
		\textbf{bounded} (or bounded above or ... ) if its range is bounded 
		(or ... ).
		\item The sequence $\{a_k\}_{k = 1}^\infty$ \textbf{converges} to the 
		\textbf{limit} $a$ if for each $\epsilon > 0$ there is a positive 
		integer $K$ such that $|a_k - a| < \epsilon$ for all $k \ge K$. 
		\item A \textbf{subsequence} $\{a_{k_j}\}_{j = 1}^\infty$ of a 
		sequence $\{a_k\}_{k = 1}^\infty$ is the composition of 
		$\{a_k\}_{k = 1}^\infty$ with an increasing sequence 
		$\{k_j\}_{j = 1}^\infty$ of integers 
	\end{enumerate}
\end{defn}

\begin{pblm}%10
	Let $\{a_k\}_{k = 1}^\infty$ be a non-decreasing sequence of real numbers. Then 
	$\{a_k\}_{k = 1}^\infty$ is bounded above iff $\{a_k\}_{k = 1}^\infty$ converges
	to the least upper bound of the set $\{a_k : k \in \Zp\}$. (Of course 
	there is a symmetric result for non-increasing sequences.)
\begin{proof}
	Let $\{a_k\}_{k=1}^\infty$ be bounded above. Now, let $L$ be the least upper bound of 
	the set $\{a_k: k \in \Zp\}$. Since $\{a_k\}$ is a non-decreasing sequence, 
	for each $ k \in \Zp$, $a_k \le a_{k + 1} \le a_{k + 2} \le ... $ 
	And also, for all $k \in \Zp$, $a_k \le L$ since $L$ is the least upper bound. 
	Therefore for all $k \in \Zp$, 
	\begin{equation*}
		a_k \le a_{k + 1} \le \cdots \le L
	\end{equation*}
	For any $\epsilon > 0$, let $a^\ast = L - \epsilon < L$. Then since $L$ is the least 
	upper bound of the set $\{a_k: k \in \Zp\}$, we know that $a^\ast \le a_j$ for 
	some $j \in \Zp$. Therefore (using the fact that the sequence is 
	non-decreasing) for all $k \ge j$, 
	\begin{equation*}
		L \ge a_k \ge L - \epsilon 
	\end{equation*}
	which can be re-written as 
	\begin{equation*}
		0 \le |L - a_k| \le \epsilon
	\end{equation*}
	Thus $\{a_k\}_{k=1}^\infty$ converges to the least upper bound 
	

	Now, if $\{a_k\}_{k=1}^\infty$ converges to the L.U.B $L$ of $\{a_k:k\in\Zp\}$
	and $a_k \le a_{k+1}$ for all $k \in \Zp$, then for all $k \in \Zp$, 
	we have 
	\begin{equation*}
		a_k \le a_{k + 1} \le \cdots \le L 
	\end{equation*}
	Thus $\{a_k\}_{k = 1}^\infty$ is bounded above. 
\end{proof}
\end{pblm}

\pagebreak
\begin{rmk}%11
~
\begin{enumerate}
	\item Note the distinction between the sequence $\{a_k\}_{k = 1}^\infty$, which 
	as a function is a set of ordered pairs of real numbers, and the set 
	$\{a_k: k \in \Zp\} \subset \R$ which is the range of that 
	function. 
	\item If $\{a_k\}_{k = 1}^\infty$ is non-decreasing and not bounded above, 
	we often write $\lim\limits_{k\to\infty} a_k = \infty$ and speak as though 
	$\infty$ were a number ``way out there at the end of the number line.'' We do 
	not, however, do arithmetic with $\infty$. 
\end{enumerate}
\end{rmk}

\begin{defn}\label{d:convergesequence}%12
~
\begin{enumerate}
	\item $\{a_k\}_{k = 1}^\infty$ be a sequence of real numbers. The numbers 
	$s_n = \sum\limits_{k = 1}^n a_k$ are the \textbf{partial sums of the infinite 
	series} $\sum\limits_{k = 1}^\infty a_k$. The series \textbf{converges} if 
	$\lim\limits_{n\to\infty}s_n$ exists (as a real number). Otherwise the series 
	\textbf{diverges}. If the series converges, the number $s = \lim\limits_{n\to\infty}
	s_n$ is called the \textbf{sum of the infinite series} and is denoted 
	$\sum\limits_{k = 1}^\infty a_k$. 
	\item More generally, if $S$ is any set of positive integers, and for any 
	positive integer $n$, $S_n = \{k \in \Zp:k\in S \text{ and }k\le n\}$, 
	we define the finite sums $s_n = \sum\limits_{k \in S_n} a_k$ to be the 
	\textbf{partial sums} of the series $\sum\limits_{k \in S}a_k$, and say that 
	the series \textbf{converges} if $\lim\limits_{n\to\infty}s_n$ exists (as a 
	real number). Otherwise the series \textbf{diverges}. If the series converges, 
	the number $s = \lim\limits_{n\to\infty}s_n$ is called the \textbf{sum of the 
	series} and is denoted $\sum\limits_{k \in S}a_k$. 
\end{enumerate}
\end{defn}

\begin{pblm}%13
	Let $\{a_k\}_{k = 1}^\infty$ be a sequence of non-negative reals. Let 
	$\{F_j\}_{j = 1}^\infty$ be any nested sequence of sets of positive integers 
	such that $F_1 \subset F_2 \subset ...$ and $\bigcup\limits_{j = 1}^\infty F_j 
	= \Zp$. For each $j$ let $f_j = \sum\limits_{k \in F_j} a_k$. Then 
	$\sup_j f_j = \sum\limits_{k = 1}^\infty a_k$. (This means that if either 
	number is finite, then they both are and and then they are equal. Notice 
	that it is not assumed that the sets $F_j$ are finite, so the $f_j$ need not 
	be finite.)\\
	%\textbf{Remark}: This is a strong form of saying that a series with non-negative 
	%terms can be summed in any order without affecting the sum. 
\begin{proof}

$\sum\limits_{k=1}^\infty a_k = \sum\limits_{k \in \cup_{j=1}^\infty F_j}a_k \in \{f_j\}_{j=1}^\infty$
And so, by definition 
\begin{equation*}
	\sum\limits_{k=1}^\infty a_k ~~\le~~ \sup_j f_j
\end{equation*}

Now, since $\forall j \in \Zp$, $F_j \subset F_{j+1}$, then 
\begin{equation*}
f_j = \sum\limits_{k \in F_j}a_k \le \sum\limits_{k\in F_{j+1}} a_k = f_{j+1}
\end{equation*}
Now, if the $F_j$ are finite, then 
using the fact that $\{f_j\}_{j=1}^\infty$ is a sequence of non-decreasing positive 
real values, we have 
\begin{equation*}
	f_j = \sum\limits_{k\in F_j}a_k \le \lim\limits_{n\to\infty}f_n = \sum\limits_{k=1}^\infty a_k ~~~ \forall j \in \Zp
\end{equation*}
On the other hand, if the $F_j$ are not necessarily finite, the values can be re-ordered in 
increasing order, and then each $f_j$ can be written 
as the limit of a sequence of partial sums 
$\lim\limits_{n\to\infty}\sum\limits_{k=1}^n a_{j(k)}$ 
where $F_j = \{j(k)\}_{k=1}^\infty$. Then all of the partial sums are bounded above by 
$\sum\limits_{k=1}^\infty a_k$, and so $f_j \le \sum\limits_{k=1}^\infty a_k$ for all $j$ 
and irrespective whether the $F_j$ are finite or infinite. 

Therefore $\sum\limits_{k=1}^\infty a_k$ is an upper bound of $\{f_j\}_{j=1}^\infty$, and so 
\begin{equation*}
	\sup_j f_j \le \sum\limits_{k\in F_j}a_k 
\end{equation*}
\end{proof}
\end{pblm}

\begin{pblm}%14
	Consider the double indexed set $\{a_{j,k}\}_{j,k=1}^\infty$ of non-negative 
	numbers. Set $S_n = \sum\limits_{j,k=1}^na_{j,k}$. (This means the sum of the 
	$n^2$ terms $a_{j,k}$ where $1 \le j \le n$ and $1 \le k \le n$.) Let 
	$A_1 = \lim\limits_{n\to\infty}S_n$, 
	$A_2 = \sum\limits_{j=1}^\infty \sum\limits_{k=1}^\infty a_{j,k}$, 
	$A_3 = \sum\limits_{k=1}^\infty \sum\limits_{j=1}^\infty a_{j,k}$. 
	Then $A_1=A_2=A_3$. This means that if any 
	one of these three numbers is finite, then all of them are and they are 
	equal. (This is, of course, the equivalent for series of the fact that a 
	double integral can be evaluated as an iterated integral. For instance, $A_2$ 
	if it exists, is the limit as $n \to \infty$ of 
	$\sum\limits_{j=1}^\infty\sum\limits_{k=1}^\infty a_{j,k}$.)
\begin{proof}
	Now, as a sum of non-negative numbers for any $n$,
	\begin{equation*}
		S_n = \sum\limits_{j=1}^n\sum\limits_{k=1}^na_{j,k}\le
		\sum\limits_{j=1}^\infty\sum\limits_{k=1}^\infty a_{j,k}=A_2
	\end{equation*}
	which implies that, since this is true for any $n$, $A_1 \le A_2$. 

	Now to show that $A_2 \le A_1$, we will consider the sequence 
	$\{T_n\}_{n=1}^\infty = \{\sum\limits_{j=1}^n\sum\limits_{k=1}^\infty a_{j,k}\}_{n=1}^\infty$. This is a 
	sequence of non-decreasing positive values with $\lim\limits_{n\to\infty}T_n = A_2$. 
	Also for all $n \in \Zp$, 
	\begin{equation*}
		\sum\limits_{j=1}^n\sum\limits_{k=1}^\infty a_{j,k} \le A_1
	\end{equation*}
	And so $A_2 \le A_1$. 

	Note that $S_n = \sum\limits_{j=1}^n\sum\limits_{k=1}^na_{j,k}=\sum\limits_{k=1}^n\sum\limits_{j=1}^na_{j,k}$
	As a finite sum, and that, since this is true for all $n$, then showing that $A_1 = A_2$ is equivalent to 
	showing $A_1 = A_3$

\end{proof}
\end{pblm}

\begin{defn}%15
~
	\begin{enumerate}[(a)]
		\item An \textbf{open ball} about a real number $x$ of radius $r$ 
		is the set $B_r(x) = (x - r, x + r)$. 
		\item A set $G$ of real numbers is \textbf{open} if it contains an 
		open ball about each of its points. 
		\item A set $F$ of real numbers is \textbf{closed} if every cluster 
		point of $F$ is contained in $F$. 
	\end{enumerate}
\end{defn}

\begin{pblm}%16
	Every open subset $G$ of real numbers can be expressed as a countable union 
	of pairwise disjoint open intervals. (Think about the rationals in $G$.)
\begin{proof}
Consider all of the the rational numbers in $G$. This is a countable set, so 
indexing by any subset of this set will also be countable. 

Now, for all rational $x \in G$, let $I_x$ be the largest open interval in $G$ containing 
$x$. For any $y \in G$ such that $y$ is not rational, Because $G$ is open, we can find 
$\epsilon > 0$ such that $y \in (x - \epsilon, x + \epsilon)\subset G$ for some rational 
$x \in G$. Then the set $\{I_x : x \in G \text{ is rational} \}$ is an open cover of $G$. 

To find a subset that are pairwise disjoint and that covers all of $G$, consider any two 
rational numbers $x, y \in G$. If $I_x \cap I_y \neq \emptyset$, then by the maximality 
of $I_x$ and $I_y$ $x \in I_x \cup I_y \subset G$, and so $I_x = I_x \cup I_y = I_y$. 
Thus every pair of intervals
$I_x, I_y$ is either disjoint or identical intervals. 
\end{proof}
\end{pblm}

\begin{defn}\label{d:compact}%17
	The set $E$ of real numbers is \textbf{compact} if every sequence 
	$\{a_k\}_{k=1}^\infty$ of elements of $E$ has a subsequence that converges 
	to an element of $E$. 
\end{defn}

\begin{fact}%18
	These properties are equivalent for a set $E$ of real numbers. 
	\begin{enumerate}[(a)]
		\item $E$ is compact. 
		\item $E$ is closed and bounded. 
		\item Every infinite subset of $E$ has a cluster point in $E$. 
		\item If $\{G_\lambda: \lambda \in \Lambda\}$ is a collection of open 
		sets such that $E \subset \bigcup\limits_{\lambda \in \Lambda}G_\lambda$ 
		(an \textbf{open cover} of $E$), then some finite subcollection is 
		also an open cover of $E$.
	\end{enumerate}
\end{fact}



