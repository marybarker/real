\chapter{Measure on $\R$}

\begin{rmk}%54
	There are several ways that we could develop a theory of measure on the 
	entire real line. One is to start over and just repeat the development we 
	have already been through. (Actually we could just have started that way.) 
	The only important difference is that now some sets will have infinite 
	measure ($\R$ itself for instance), so that we would have to keep in mind 
	that the infinite sums are not guaranteed to converge. What we will 
	actually do is use our theory on $[0,1]$ to create a theory on $\R$. 
\end{rmk}

\begin{defn}\label{d:measurableset}%55
	\begin{enumerate}[(a)] 
	\item For any real number $x$, let $\M_x$ be the collection of all subsets 
	$E$ of $[x,x+1]$ such that $E-x$ is a measurable subset of $[0,1]$. For 
	$E \in \M_x$ we set $m(E) = m(E - x)$. It is apparent that $\M_x$ is a 
	$\sigma$-algebra of subsets of $[x,x+1]$ on which all the properties of 
	\#44 hold. 
	\item For any subset $E$ of $\R$, and each integer $n$, let $E_n = 
	E \cap [n,n+1)$, We say that $E$ is Lebesgue measurable if $E_n \in \M_n$ 
	for each $n$. We denote the collection of all Lebesgue measurable subsets 
	of $\R$ by $\M_\R$. If $E$ is measurable, we define the Lebesgue measure 
	of $E$ by $m(E) = \sum\limits_{n=-\infty}^\infty m(E_n)$.  
	\end{enumerate}
\end{defn}

\begin{rmk}%56
	It is very easy to verify that the set $\M_\R$ of Lebesgue measurable 
	subsets of $\R$ is a $\sigma$-algebra of sets containing the Borel sets, 
	and that the Lebesgue measure $m$ has all the properties of \#44 except 
	that we may now interpret translation invariance (property 4) as ordinary 
	translation invariance instead of ``translation invariance mod 1,'' that 
	is, for any $E \in \M_\R$, and any real number $c$, $E + c \in \M_\R$ and 
	$m(E) = m(E+c)$. 

	Of course we do now have the possibility that $m(E) = \infty$. For this 
	purpose we regard $\infty$ as just another number, more or less. It is 
	necessary however, to avoid expressions like $\infty - \infty$, so at times 
	we will need to be careful about whether the measure of a set is finite. 
\end{rmk}

\begin{pblm}\label{p:finitemeasurable}%57
	if $E$ is a measurable set of finite measure, then for any $\epsilon > 0$ 
	there is a finite collection of open intervals $\{G_k\}_{k=1}^n$ so that 
	\begin{equation*}
		m\left(E\setminus\bigcup\limits_{k=1}^nG_k\right) + 
		m\left(\bigcup\limits_{k=1}^nG_k\setminus E\right) < \epsilon
	\end{equation*}
	(Thus, every measurable set is ``almost a finite union of open sets.'')
\begin{proof}
	By 53, there is a closed set $F \subset E$ such that 
	$m(E \setminus F) < \frac{\epsilon}{4}$
	Now, define $F_n = F \cap [-n,n]$. Then each $F_n$ is compact, bounded by $E$ 
	and 
	\begin{equation*}
		F_1 \subset F_2 \subset \dots
	\end{equation*}
	Furthermore, $m(F) = \sum\limits_{n=1}^\infty m(F_n \setminus F_{n-1})$ converges, 
	so there is $N$ such that 
	\begin{equation*}
		\sum\limits_{n=N+1}^\infty m(F_n\setminus F_{n-1}) < \frac{\epsilon}{4}
	\end{equation*}
	So $m(E \setminus F_N) = m(E \setminus F) + m(F \setminus F_N) < \frac{\epsilon}{2}$. 

	Since $F_N$ is measurable, there is an open $G$ such that $F_N \subset G$, which 
	we can write as a union of open intervals $G = \bigcup\limits_{k=1}^\infty G_k$. 
	Since $F_N$ is compact, there is a finite subcover of $G$ such that 
	\begin{equation*}
		F_N \subset \bigcup\limits_{k=1}^nG_k
	\end{equation*}

	Therefore since $F_N \subset \bigcup\limits_{k=1}^nG_k$, 
	\begin{equation*}
		m\left(E \setminus \bigcup\limits_{k=1}^nG_k\right) \le m(E \setminus F_N) < \frac{\epsilon}{2}
	\end{equation*}

	Also, since $F_N \subset \bigcup\limits_{k=1}^nG_k$, 
	\begin{equation*}
		m\left(\bigcup\limits_{k=1}^nG_k\setminus F_N\right) \le 
		m\left(G\setminus E\right) < \frac{\epsilon}{2}
	\end{equation*}
	and since $F_N \subset E$, 
	\begin{equation*}
		m\left(\bigcup\limits_{k=1}^n G_k \setminus E\right) < 
		m\left(G\setminus F_N\right) < \frac{\epsilon}{2}
	\end{equation*}
	and so putting the two results together we have 
	\begin{equation*}
		m\left(E\setminus\bigcup\limits_{k=1}^nG_k\right) + 
		m\left(\bigcup\limits_{k=1}^nG_k\setminus E\right) < 
		\frac{\epsilon}{2}+\frac{\epsilon}{2}=\epsilon
	\end{equation*}
\end{proof}
\end{pblm}

