\chapter{Measurable Functions}
\begin{defn}%58
	A real-valued function $f$ defined on a measurable subset $E$ of $\R$ is 
	\textbf{measurable} if for every real number $\alpha$, 
	$\{x \in E: f(x) < \alpha\}$ is measurable. 
\end{defn}

\begin{pblm}%59
	These are equivalent: 
	\begin{enumerate}[(a)]
	\item $f$ is measurable %a
	\item for every $\alpha$, $\{x \in E: f(x) \ge \alpha\}$ is measurable %b
	\item for every $\alpha$, $\{x \in E: f(x) > \alpha\}$ is measurable %c
	\item for every $\alpha$, $\{x \in E: f(x) \le \alpha\}$ is measurable %d
	\item for every open set $G$, $\{x \in E: f(x) \in G\}$ is measurable. %e
	(Show (a) $\leftrightarrow$ (b) $\Rightarrow$ (c) $\leftrightarrow$ (d) 
	$\Rightarrow$ (a) and (a) \& (c) $\Rightarrow$ (e), (e) $\Rightarrow$ (a))
	\end{enumerate}
\begin{proof}
	\noindent $(a) \Leftrightarrow (b)$ Let $f$ be a measurable function. Then for 
	any $\alpha$, the set $\{x \in E: f(x) < \alpha\}$ is measurable. Since the 
	complement of a measurable set is measurable, this means that $\{x \in E: f(x) \ge \alpha\}$ 
	is measurable for any $\alpha$. 

	\noindent $(b) \Rightarrow (c)$ 
	If $f$ is measurable, then for every $\alpha$, 
	the set $\{x \in E: f(x) \ge \alpha\}$ is measurable. Consider the sets 
	\begin{equation*}
		\left\{x \in E: f(x) \ge \alpha + \frac{1}{i}\right\}_{i=1}^\infty
	\end{equation*}
	Each of these sets is measurable, and so as the countable union of 
	measurable sets is measurable, this means that t
	\begin{equation*}
		\bigcup\limits_{i=1}^\infty \left\{x \in E : f(x) \ge \alpha + \frac{1}{i}\right\} 
		= \{x \in E: f(x) > \alpha\}
	\end{equation*}
	is measurable.

	\noindent $(c) \Leftrightarrow (d)$ Given that for all $\alpha \in \R$, 
	$\{x \in E: f(x) > \alpha\}$ is measurable, then we also have the complement 
	of a measurable set is measurable. Therefore for 
	all real $\alpha$, $\{x \in E: f(x) \le \alpha\}$ is measurable.  

	\noindent $(d) \Rightarrow (a)$ 
	Similarly to showing $(b)$ implies $(c)$, consider the sets
	\begin{equation*}
		\left\{x \in E: f(x) \le \alpha - \frac{1}{i}\right\}_{i = 1}^\infty
	\end{equation*}
	As each set is measurable, the countable union is also, and so 
	\begin{equation*}
		\bigcup\limits_{i=1}^\infty\left\{x \in E: f(x) \le \alpha - \frac{1}{i}\right\} = 
		\{x \in E: f(x) < \alpha\}
	\end{equation*}
	is measurable for any $\alpha$. Thus $f$ is a measurable function on $E$.

	\noindent $(a),(c) \Rightarrow (e)$ For any open set $G$, we can write 
	$G$ as a union of open intervals about each of its points. Say, $G = \bigcup\limits_{i=1}^\infty F_i$ 
	where each $F_i$ is an open interval. If the number of points in $G$ is some finite $n \in \Zp$, 
	we can take $F_{n+1}, F_{n+2} ... = \emptyset$. 

	Now, for each $F_k = (a_k, b_k)$, since $f$ is measurable, we know that the intersection of 
	the sets $\{x \in E: f(x) < b_k\}$ and $\{x \in E: f(x) > a_k\} = \{x \in E: f(x) \in F_k\}$ 
	is measurable. Therefore, since the countable union of measurable sets (each of which is 
	the intersection of two measurable sets) is measurable, then
	\begin{equation*}
		\{x \in E: f(x) \in G\} = \bigcup\limits_{i=1}^\infty \{x \in E: f(x) \in F_i\}
	\end{equation*}
	is measurable. 

	\noindent $(e) \Rightarrow (a)$ Now, assume that for all open $G$, 
	the set $\{x \in E: f(x) \in G\}$ is measurable. For any $\alpha \in \R$, 
	we can write 
	\begin{equation*}
		\begin{array}{rcl}
		\{x \in E: f(x) < \alpha\} & = & 
		\left\{x \in E: f(x) \in \bigcup\limits_{k=1}^\infty (\alpha - k, \alpha - (k - 1))\right\} \\
		& = & \bigcup\limits_{k=1}^\infty\left\{x \in E: f(x) \in (\alpha - k, \alpha - (k - 1))\right\}
		\end{array}
	\end{equation*}
	which, as a countable union of measurable sets is measurable. Therefore $f$ is measurable. 
\end{proof}
\end{pblm}

\begin{ex}%60
~
\begin{enumerate}
	\item if $E$ is a measurable set, the \textbf{characteristic function} 
	of $E$ is defined by 
	\begin{equation*}
		\chi_E (x) = 
	\left\{ \begin{matrix} 
		1, & x \in E, \\
		0, & x \notin E
	\end{matrix} \right.
	\end{equation*}
	is measurable. 
	\item By part (e) of the previous problem, any continuous function with a 
	measurable domain is measurable. (What property of continuous functions 
	does this follow from?) 
\end{enumerate}
\end{ex}

\pagebreak
\begin{pblm}%61
	If $f$ is measurable with domain $E$, and $c \in \R$, then $cf$ and 
	$f + c$ are measurable. 
\begin{proof}
	If $c > 0$, then for any $\alpha \in \R$, $\alpha /c \in \R$, and 
	\begin{equation*}
		\{x \in E: cf(x) < \alpha\} = \{x \in E: f(x) < \alpha/c\}
	\end{equation*}
	is measurable, since $f$ is measurable. Therefore $cf$ is measurable for all 
	$c > 0$. 

	If $c = 0$, then for any $\alpha \in R$, 
	the set $\{x \in E: cf(x) < \alpha\}$ is either $\R\cap E$ 
	(if $\alpha \ge 0$), which is measurable, or else $\emptyset$ (if $\alpha < 0$), 
	which is also measurable. 

	If $c < 0$, then for any $\alpha \in \R$, the set $\{x \in E: cf(x) < \alpha\} = 
	\{x \in E: f(x) \ge \alpha / c\}$ which is measurable since $f$ is measurable. 

	Thus $cf(x)$ is measurable for all $c \in \R$.

	Now, if $f$ is measurable over $E$, then for any $c \in \R$, for all $\alpha \in \R$, 
	\begin{equation*}
		\{x \in E: f(x) + c < \alpha\} = \{x \in E: f(x) < \alpha - c\}
	\end{equation*}
	which is measurable since $f$ is measurable. 

	Therefore $f(x) + c$ is measurable. 
\end{proof}
\end{pblm}

\begin{pblm}%62
	If $f$ and $g$ are measurable with domain $E$, then $f + g$ is measurable. 
	(Show $\{x \in E: f(x) + g(x) < \alpha\} = \bigcup\limits_{q} 
	(\{x \in E: f(x) < \alpha - q\} \cap \{x \in E: g(x) < q\})$ where the 
	union is over all of $q \in \Q$.)

\begin{proof}
	For each $q \in \Q$, the sets $\{x \in E: f(x) < \alpha - q\}$ and 
	$\{x \in E: g(x) < q\}$ are both measurable, since $f$ and $g$ are 
	both measurable functions. Also, as the intersection of two measurable 
	sets is measurable, for any $\alpha \in \R$ and any $q \in \Q$, 
	\begin{equation*}
		\{x \in E: f(x) < \alpha - q\} \cap \{x \in E: g(x) < q\}
	\end{equation*}
	is measurable.  Also, since the rationals are countable, the union 
	\begin{equation*}
		\bigcup\limits_{q \in \Q}\left\{\{x \in E: f(x) < \alpha - q\} \cap \{x \in E: g(x) < q\}\right\}
	\end{equation*}
	is measurable. 
	And so 
	\begin{equation*}
		\{x \in E: f(x) + g(x) < \alpha\} = 
		\bigcup\limits_{q \in \Q}\{x \in E: f(x) < \alpha - q\} 
		\cap \{x \in E: g(x) < q\} 
	\end{equation*}
	is measurable. 
\end{proof}
\end{pblm}

\begin{pblm}\label{p:minmaxmeasurable}%63
	if $f$ and $g$ are measurable with domain $E$, then $\max\{f,g\}$ and 
	$\min\{f,g\}$ are measurable.
\begin{proof}
	For any $\alpha$, $\{x \in E: max(f, g) > \alpha\}$ is equal to 
	\begin{equation*}
		\{x \in E: f(x) > \alpha\} \cup \{x \in E: g(x) > \alpha\}
	\end{equation*}
	which, as the union of two measurable sets, is measurable. Therefore 
	$max(f, g)$ is measurable. 

	Similarly for $min(f, g)$, for any $\alpha$, 
	\begin{equation*}
		\{x \in E: min(f, g) < \alpha\} = \{x \in E: f < \alpha \} \cup \{x \in E: g < \alpha\}
	\end{equation*}
	is also measurable. 
\end{proof}
\end{pblm}

\begin{defn}\label{d:pos_neg_fun}%64
	If $f$ is measurable with domain $E$, the \textbf{positive part} of $f$ is 
	the function $f_+ = \max\{f,0\}$, and the \textbf{negative part} of $f$ is 
	the function $f_- = \max\{-f,0\}$. 
\end{defn}

\begin{rmk}%65
	Note that $f = f_+ - f_-$ and $|f| = f_+ + f_-$. It follows from the 
	previous problems that if $f$ is measurable, then so are all of $f_+, f_-$, 
	and $|f|$. 
\end{rmk}

\begin{pblm}%66
	If $f$ and $g$ are measurable with domain $E$, then $f^2$ and $fg$ are 
	measurable. \\ (Do $f^2$ first. Then 
	$fg = \frac{1}{2}\left( (f + g)^2 - f^2 - g^2\right)$.)
\begin{proof}
	If $f$ is a measurable function, then for any $\alpha$, if $\alpha \ge 0$, 
	\begin{equation*}
		\{x \in E: [f(x)]^2 < \alpha\} = \{x \in E: f(x) < \sqrt{\alpha}\} \cap \{x \in E: f(x) > -\sqrt{\alpha}\}
	\end{equation*}
	as the intersection of two measurable sets, is measurable. 
	If on the other hand, $\alpha < 0$, then $\{x \in E: f^2 < \alpha\} = \emptyset$, which is measurable. 
	Therefore for any $\alpha$, $\{x \in E: [f(x)]^2 < \alpha\}$ is measurable, and so $f^2$ is measurable. 

	Now, as we have seen, the linear combination of measurable functions is measurable, and the 
	square of a measurable function is measurable. Therefore 
	\begin{equation*}
		fg = \frac{1}{2}\left(f^2 - f^2 + 2fg + g^2 - g^2\right) = 
		\frac{1}{2}\left((f+g)^2 - f^2 - g^2\right)
	\end{equation*}
	is measurable. 
\end{proof}
\end{pblm}

\begin{defn}\label{d:ae}%67
	We say that a property holds \textbf{almost everywhere} (abbreviated 
	\textbf{a.e.}) if the set where it does not hold has measure zero. In this 
	context a set of measure zero is often called a \textbf{null set}. We also 
	say that a property that holds in a set $E$ except on a null set holds for 
	\textbf{almost all} $x$ in $E$. 
\end{defn}

\begin{pblm}%68
	If $f$ and $g$ are defined on a measurable set $E$, $f$ is a measurable 
	function, and $f = g$ a.e., then $g$ is measurable. 
\begin{proof}
	For all $\alpha$, the sets $\{x \in E: f(x) < \alpha\}$ and 
	$\{x \in E: f(x) \neq g(x)\}^c$ are measurable. Therefore 
	\begin{equation*}
		\{x \in E: f(x) < \alpha\} \cap \{x \in E: f(x) \neq g(x)\}^c
	\end{equation*}
	is measurable, and 
	\begin{equation*}
		\underbrace{ 
			\overbrace{\{x \in E: f(x) < \alpha\}\cap\{x \in E: f(x) \neq g(x)\}^c}^{\text{measurable}} 
			\cup \overbrace{\{x \in E: f(x) \neq g(x) \& g(x) < \alpha \}}^{\text{measure }0}
		}_{\{x \in E: g(x) < \alpha\}}
	\end{equation*}
	is measurable. 
\end{proof}
\end{pblm}

\begin{defn}\label{d:extendedreals}%69
	The \textbf{extended real numbers} consist of $\R$ together with $+\infty$ 
	and $-\infty$. These have the property that $-\infty < r < +\infty$ for 
	every real number $r$, but we try not to do arithmetic with $\pm\infty$. It 
	is often convenient to allow measurable functions to have range in the 
	extended real numbers. The preceding problem shows that as long as 
	functions take finite values almost everywhere, we can do arithmetic with 
	them without bothering about what happens on a null set. 
\end{defn}

\begin{pblm}%70
	If $f$ is a measurable extended real-valued function defined on a bounded 
	measurable set $E$ and $f$ is finite a.e., then for any $\epsilon > 0$ 
	there is $M$ so that $|f| \le M$ except on a set of measure less than 
	$\epsilon$. (So any measurable function on a bounded set is ``almost 
	bounded.'') \\ (Remember \mPref{p:osouseful})
\begin{proof}
	Let $E_k = \{x \in E: |f(x)| > k\}$. Each of these sets is 
	measurable, and $m(E_{k+1}) \ge m(E_k) \ge ...$ for all $k$. Therefore 
	for any $\epsilon > 0$, there is some $k$ such that 
	\begin{equation*}
		m(E_k) < \epsilon
	\end{equation*}
	which implies that for all $x \notin E_k$, 
	\begin{equation*}
		|f(x)| \le k
	\end{equation*}
\end{proof}
\end{pblm}

\begin{pblm}%71
	Is it necessary to assume in the previous problem that the set $E$ is 
	bounded? 

	We need the measure of each of those $E$ not to be infinite 
	in the previous example, else \mPref{p:osouseful} need not apply. 
\end{pblm}

\begin{pblm}%72
	If $\{f_n\}_{n=1}^\infty$ is any sequence of real-valued measurable 
	functions defined on a measurable set $E$, then the function 
	$F(x) = \sup_n f_n(x)$ is a measurable. (Note, for instance, that 
	$F(x) = \infty$ if $f_n(x) \rightarrow \infty$ as $n \rightarrow \infty$.)
\begin{proof}
	for any $\alpha$, 
	\begin{equation*}
	\begin{array}{rcl}
		\{x \in E: F(x) > \alpha\} &=& \{x \in E: \sup \{f_n(x)\}_{n=1}^\infty > \alpha\}\\
		&=& \bigcup\limits_{n=1}^\infty\{x\in E: f_n(x) > \alpha\} \in \M
	\end{array}
	\end{equation*}
\end{proof}
\end{pblm}

\begin{defn}\label{d:limsup}%73
	The \textbf{limit superior} of a sequence of real numbers 
	$\{a_k\}_{k=1}^\infty$ is the extended real number 
	$E = \inf_n \sup_{k\ge n} a_k$. We write $A = \lim\sup a_k$. \\
	Note that the sequence $\{\sup_{k\ge n}a_k\}_{n=1}^\infty$ is 
	non-increasing, so the inf is the same as the limit of this sequence, 
	keeping in mind that all elements of this sequence may be $\infty$. 

	Similarly the \textbf{limit inferior} of $\{a_k\}_{k=1}^\infty$ is 
	$B = \sup_n\inf_{k\ge n}a_k$. We write $B = \lim\inf a_k$. The sequence 
	$\{\inf_{k\ge n}a_k\}_{n=1}^\infty$ is non-increasing, so the sup is the 
	same as the limit of this sequence. 
\end{defn}

\begin{pblm}\label{p:limeqliminfeqlimsup}%74
	For any sequence $\{a_k\}_{k=1}^\infty$ of real numbers, $\lim\sup a_k 
	\ge \lim\inf a_k$. $\lim\sup a_k = \lim\inf a_k$ if and only if the 
	sequence $\{a_k\}_{k=1}^\infty$ converges. 
\begin{proof}
	$(\Rightarrow)$ Let $\lim\sup\limits_{a_k} = \lim\inf\limits_{a_k}$. Then 
	for each $a_k$, 
	\begin{equation*}
	\begin{array}{rcccl}
		\inf\limits_{k\ge n}a_k & \le & a_k & \le & \sup\limits_{k\ge n} a_k \\
		\inf a_k - \inf a_k & \le & a_k - \inf a_k & \le & \sup a_k - \inf a_k\\
		0 & \le & a_k - \inf a_k & \le & \sup a_k - \inf a_k\\
		\lim\limits_{n\to\infty}0 & \le & \lim\limits_{n\to\infty}(a_k - \inf a_k ) 
		& \le & \lim\limits_{n\to\infty} \sup a_k - \lim\limits_{n\to\infty}\inf a_k\\
		0 & \le & \lim\limits_{n\to\infty}(a_k - \inf a_k) & \le & 0. 
	\end{array}
	\end{equation*}
	Therefore, $\lim\limits_{n\to\infty}(a_k - \inf a_k) = 0$, so 
	$\lim\limits_{n\to\infty}a_k = \lim\limits_{n\to\infty}\inf a_k = 
	\lim\limits_{n\to\infty} \sup a_k$. Therefore the limit  as $n\to\infty$ of $a_k$ 
	exists (and is a real number), so the sequence converges. 

	\noindent$(\Leftarrow)$ Assume that $\{a_k\}_{k=1}^\infty$ converges to a limit. 
	Then all of its subsequences converge to the same limit, and since both $\sup a_k$ 
	and $\inf a_k$ are subsequences of $\{a_k\}_{k=1}^\infty$, they must converge to 
	the same limit. 
\end{proof}
\end{pblm}

\begin{pblm}\label{p:ext_real_val_seq}%75
	If $\{f_n\}_{n=1}^\infty$ is any sequence of extended real-valued 
	measurable functions defined on a measurable set $E$, and if $f(x) = 
	\lim\limits_{n\to\infty}f_n(x)$ exists for each $x \in E$, then $f$ is an 
	extended real-valued measurable function on $E$. 
\begin{proof}
	Suppose $f(x) = \lim\limits_{n\to\infty} f_n(x)$ exists for each $x \in E$. Let $\alpha \in \mathbb{R}$. Then,

	\begin{align*}
		\{ x\in E : f(x) < \alpha \}
		&= \{ x \in E : \lim\limits_{n\to\infty}f_n(x) < \alpha \} \hspace{27 pt} \text{(By definition of $f$.)} \\
		&= \{ x \in E : \lim \sup f_n(x) < \alpha \} \hspace{19 pt} \text{(By 74)} \\
		&= \{ x \in E : \inf\limits_n \sup f_n(x) < \alpha \} \hspace{20 pt} \text{(By definition.)}
	\end{align*}
	
	By \mPref{p:limeqliminfeqlimsup}, $\sup f_n(x)$ is measurable, 
	and applying \mPref{p:limeqliminfeqlimsup} again, $\inf\limits_{n} \sup f_n(x)$ 
	is measurable, so $f$ is measurable.
\end{proof}
\end{pblm}

\begin{rmk}%76
	~
	\begin{enumerate}[(i)]
	\item The conclusion still holds in the previous problem if the sequence 
	$\{f_n\}_{n=1}^\infty$ converges a.e. on $E$. 
	\item That the pointwise limit of a sequence of measurable functions is 
	measurable is essential to the theory of integration that we are about to 
	create. It is the step that allows us to avoid the difficulty that the 
	pointwise limit of a sequence of Riemann integrable functions may not be 
	Riemann integrable. 
	\end{enumerate}
\end{rmk}

\begin{defn}\label{d:simplefunction}%77
	A \textbf{simple function} is a measurable function whose range is a finite 
	subset of $\R$. A simple function $s$ taking values $a_j$ on measurable 
	sets $E_j$ can be written 
	\begin{equation*}
		s = \sum\limits_{j=1}^n a_j \chi_{E_j}
	\end{equation*}
	Of course this representation is not unique. The \textbf{canonical} 
	representation of a simple function is the sum where $\{E_j\}_{j=1}^n$ 
	are disjoint and $j \neq k$ implies $a_j \neq a_k$. (In other words, each 
	$E_j = s^{-1} (a_j)$ for the distinct values $a_j$ in the range of $s$.)
\end{defn}

\pagebreak
\begin{pblm}\label{p:contalmostsimple}%78
	Let $f$ be an extended real-valued measurable function defined on an 
	interval $[a,b]$. Let $M > 0$. For any $\epsilon > 0$ there is a simple 
	function $s$ so that $|f(x) - s(x)| < \epsilon$ on the set where $|f| 
	\le M$. \\ (Partition $[-M,M]$ into slices and use the sets where $f$ 
	takes values in a slice to define $s$.)
\begin{proof}
	For any $\epsilon > 0$, let $N$ be the smallest integer such that 
	$N \ge \frac{2M}{\epsilon}$. 
	We will cover the interval $[-M,M]$ with a family of disjoint intervals 
	\begin{equation*}
		\left\{[ -M+(j-1)\epsilon, -M + j\epsilon )\right\}_{j=1}^N
	\end{equation*} 
	each of which has length 
	$\epsilon$. Let $E_j$ denote the $j^{th}$ interval, and $e_j$ the midpoint of that 
	interval. 

	Let $s(x) = \sum\limits_{j=1}^N e_j\chi_{E_j}(f(x))$. Then for every $x$, 
	$s(x)$ is the midpoint of an $\epsilon$-length interval containing $f(x)$, so 
	\begin{equation*}
		|f(x) - s(x)| \le \epsilon/2,~~\text{ and so }~~|f(x) - s(x)| < \epsilon
	\end{equation*}
\end{proof}
\end{pblm}

\begin{defn}\label{d:stepfunction}%79
	A \textbf{step function} $p$ is a simple function whose domain is a closed 
	bounded interval $[a,b]$ and which can be written in the form 
	$\sum_{j=1}^n a_j\chi_{\epsilon_j}$ where the $E_j$ are intervals. 
\end{defn}

\begin{pblm}\label{p:simplealmoststep}%80
	Let $s$ be a simple function whose domain is a closed bounded interval 
	$[a,b]$. For every $\epsilon > 0$ there is a step function $p$ so that 
	$m(\{x \in [a,b]: s(x) \neq p(x)\}) < \epsilon$. \\ (Remember \mPref{p:finitemeasurable}). 
\begin{proof}
	As a simple function, we can re-write $s$ as 
	\begin{equation*}
		s = \sum\limits_{j=1}^n a_j \chi_{E_j}
	\end{equation*}
	for some measurable sets $E_j$ and some $a_j\in \R$ such that $a_j\neq 0$. 

	For each $E_j$, let $\{G_{j_k}\}_{k=1}^{m_j}$ be a finite set of open 
	intervals such that 
	\begin{equation*}
		m^\ast\left(E \setminus \bigcup\limits_{k=1}^{m_j} G_{j_k}\right) + 
		m^\ast\left(\bigcup\limits_{k=1}^{m_j} G_{j_k} \setminus E\right) <
		\frac{\epsilon}{n}
		%\frac{\epsilon}{a_jn}. 
	\end{equation*}
	And furthermore, let $G_j = \bigcup\limits_{k=1}^{m_j} G_{j_k}$ for each $j$. Then 
	\begin{equation*}
		p = \sum\limits_{j=1}^n a_j \chi_{G_j}
	\end{equation*}
	is a step function such that for all $x \in [a, b]$, 
	\begin{equation*}
		s(x) \neq p(x)
	\end{equation*} 
	implies that 
	\begin{equation*}
		x \in N = \left(E_j \setminus G_j\right) \cup\left(G_j \setminus E_j\right)
	\end{equation*}
	for at least one $j \in \{1, ... , n\}$, and therefore 
	\begin{equation*}
		m(\{x \in [a,b] : s(x) \neq p(x)\}) = 
		m(N) < \sum\limits_{j=1}^n \frac{\epsilon}{n} = \epsilon
	\end{equation*}
\end{proof}
\end{pblm}

\begin{pblm}%81 
	If $f$ is a measurable extended real-valued function defined on a closed 
	bounded interval $[a,b]$ and $f$ is finite a.e., then for any 
	$\epsilon > 0$ there is a step function $p$ and a continuous function $g$ 
	so that $|f-p|<\epsilon$ except on a set of measure less than $\epsilon$, 
	(So any measurable function is ``almost continuous.'')\\
	(This is mostly assembling the information from the previous three 
	problems.)
\begin{proof}
	if $f$ is a measurable extended real-valued function on a closed bounded 
	interval $[a,b]$ and $f$ is finite a.e., then for any $\epsilon > 0$, we
	know by \mPref{p:contalmostsimple}, there is some simple function $s$ such that 
	\begin{equation*}
		|f(x) - s(x)| < \epsilon
	\end{equation*}
	And, since $s$ is simple, we know from \mPref{p:simplealmoststep} that 
	there is some step function $p$ such that 
	$s(x) = p(x)$ except on a set of measure less than $\epsilon$. 
	Therefore 
	\begin{equation*}
		|f(x) - p(x)| = |f(x) - s(x)| < \epsilon
	\end{equation*}
	except on a set of measure less than $\epsilon$. 
\end{proof}
\end{pblm}

\begin{pblm}\label{p:approxfunwithsteporsimple}%82
	Let $E$ be a measurable set of finite measure. Suppose that $f$ and 
	$\{f_n\}_{n=1}^\infty$ are measurable functions defined on $E$ so that 
	$f_n(x) \rightarrow f(x)$ for each $x \in E$. Then for every $\epsilon > 0$ 
	and $\delta > 0$ there is a measurable set $A \subset E$ with 
	$m(A) < \delta$ and an integer $N$ so that for all $x \notin A$ and all 
	$n \ge N$, 
	\begin{equation*}
		|f_n(x) - f(x)| < \epsilon.
	\end{equation*}
	(Consider the sets $G_n = \{x: |f_n(x) - f(x)|\ge\epsilon\}$ and the 
	nested collection 
	\begin{equation*}
		E_N = \bigcup\limits_{n=N}^\infty G_n = 
		\{x: |f_n(x) - f(x)| \ge \epsilon \text{ for some }n \ge N\}. 
	\end{equation*}
	Show $\bigcup\limits_{N=1}^\infty E_N = \emptyset$.)
\begin{proof}
	since $f_n \rightarrow f$ for all $x$, then for any $\epsilon > 0$ and 
	any $\delta > 0$, there exist $A \in \M$ and $N \in \Zp$ such that 
	for all $x \notin A$ and all $n \ge N$, 
	\begin{equation*}
		|f_n(x) - f(x)| < \epsilon. 
	\end{equation*}
	Consider the sets $G_n = \{x: |f_n(x) - f(x) | \ge \epsilon$ for all $n$.
	Then 
	\begin{equation*}
		E_N = \bigcup\limits_{n=N}^\infty G_n = 
		\{x : |f_n(x) - f(x)| \ge \epsilon \text{ for some }n \ge N\} 
	\end{equation*}
	are a sequence of nested sets that, as countable 
	unions of measurable sets, are also measurable. 
	\begin{equation*}
		E_N \supset E_{N+1} \supset \dots
	\end{equation*}
	and $\lim\limits_{n\to\infty}|f_n(x) - f(x)| = 0$ for each $x$, so 
	\begin{equation*}
		\bigcap\limits_{N=1}^\infty E_N = \emptyset
	\end{equation*}
	and so $\forall \delta > 0$, there is some $N$ such that $m(E_N) < \delta$ 
	and 
	\begin{equation*}
		E_N = \{x: |f_n(x) - f(x)| \ge \epsilon \text{ for some } n \ge N\}
	\end{equation*}
	which implies that for all $x \notin E_N$, 
	\begin{equation*}
		|f_n(x) - f(x)| < \epsilon ~~\forall n > N.
	\end{equation*}
\end{proof}
\end{pblm}

\begin{rmk}%83
	It is clear that the conclusion of the preceding problem continues to hold 
	if we assume only that $f_n(x) \rightarrow f(x)$ a.e. on $E$. 
\end{rmk}

\begin{defn}\label{convergeuniformly}%84
	A sequence $\{f_n\}_{n=1}^\infty$ of functions \textbf{converges uniformly} 
	to a function $f$ on a set $E$ if for every $\epsilon > 0$ there is an 
	integer $N$ so that $|f_n(x) - f(x)| < \epsilon$ for all $n \ge N$ and 
	all $x \in E$. 

	Of course the ``uniform'' part is that $N$ does not depend on $x$. Note 
	that the preceding problem has a conclusion that somewhat resembles uniform 
	convergence, except that the set $A$ may depend on $\epsilon$. We fix this 
	in the next problem. 
\end{defn}

\pagebreak
\begin{pblm}\label{p:egoroff}%85 
	Let $E$ be a measurable set of finite measure. Suppose that $f$ and 
	$\{f_n\}_{n=1}^\infty$ are measurable functions defined on $E$ so that 
	$f_n(x) \rightarrow f(x)$ a.e. on $E$. Then for each $\alpha > 0$ there is 
	a set $A \subset E$ with $m(A) < \alpha$ and $f_n \rightarrow f$ uniformly 
	on $E\setminus A$. \\ (Apply the preceding problem repeatedly with 
	$\epsilon_n = 1/n$ and $\delta_n = 2^{-n}\alpha$.)
\begin{proof}
	Let $\alpha > 0$. Then let $\epsilon_n = \frac{1}{n}$, $\delta_n = \frac{\alpha}{2^n}$. 
	Using \mPref{p:approxfunwithsteporsimple} for each $n$, there exists a measurable 
	set $A_n \subset E$ such that $m(A_n) < \frac{\alpha}{2^n}$ and a positive 
	integer $N_n$ such that for all $x \in E\setminus A_n$ and all $j \ge N_n$, 
	\begin{equation*}
		|f_j(x) - f(x)| < \frac{1}{n}. 
	\end{equation*}
	Let $A = \bigcup\limits_{i=1}^\infty A_i$. Then 
	\begin{equation*}
		m(A) \le \sum\limits_{i=1}^\infty A_k < \sum\limits_{i=1}^\infty\frac{\alpha}{2^i} = \alpha \sum\limits_{i=1}^\infty \frac{1}{2^i} = \alpha
	\end{equation*} 
	So $m(A) < \alpha$. 

	Let $\epsilon > 0$. Then there exists some $n$ such that $\frac{1}{n} < \epsilon$ 
	(by the Archimedean property), so there exists some $N_n$ such that 
	\begin{equation*}
		|f_j(x) - f(x)| < \epsilon_n < \epsilon ~~\forall j \ge N_n \forall x \in E\setminus A_n. 
	\end{equation*}
	This is therefore true for all $x \in E\setminus A$ as $E\setminus A$ is a 
	subset of $E \setminus A_n$. 
\end{proof}
\end{pblm}

