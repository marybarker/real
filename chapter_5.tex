\chapter{The Lebesgue Integral}
\begin{rmk} ~ %86
	\begin{enumerate}
	\item Recall that in the Riemann theory of integration, for a bounded 
	function $f$ on a closed interval $[a,b]$ we define the \textbf{upper 
	integral}, $U$, and \textbf{lower integral}, $L$ as 
	\begin{equation*}
		U = \inf\left\{\int\limits_a^b s(x)dx\right\} ,~~ 
		L = \sup\left\{\int\limits_a^b r(x)dx\right\}
	\end{equation*}
	where the inf and sup are over the set of all step functions $s$ such that 
	$s \ge f$ and the set of all step functions $r$ such that $r \le f$ 
	respectively and we define the integral of a step function $s(x) = 
	\sum\limits_{j=1}^n c_j\chi_{[a_j,b_j]}(x)$ (the canonical representation) 
	to be $\int_a^b s(x) dx = \sum\limits_{j=1}^nc_j(b_j-a_j)$.(This is the 
	usual definition expressed in an unusual way. The integrals of the step 
	functions are usually called upper sums and lower sums.)

	The function $f$ is \textbf{Riemann integrable} if $U = L$. and in that 
	case we call the common value the \textbf{Riemann integral} of $f$ over 
	$[a,b]$, denoted $\int_a^bf(x)dx$. 
	\item Of course one requirement for the Lebesgue integral is that any 
	Riemann integrable function should also be Lebesgue integrable, and that 
	the values of the two integrals should be the same. Another practical 
	requirement is that there should be enough Lebesgue integrable functions 
	that are not Riemann integrable to make it worthwhile to go to the trouble 
	of developing the Lebesgue integral. 
	\end{enumerate}
\end{rmk}

\begin{defn}\label{d:integral} ~ %87
\begin{enumerate}
	\item If $s$ is a simple function with canonical representation 
	$s = \sum\limits_{j=1}^n c_j\chi_{E_j}$ 
	(recall Def \mDref{d:simplefunction}) that vanishes outside a set of finite 
	measure, the integral of $s$ is 
	\begin{equation*}
		\int s = \sum\limits_{j=1}^n c_j m(E_j).
	\end{equation*}
	\item If $E$ is any measurable set, the integral of $s$ over $E$ is 
	\begin{equation*}
		\int_E s = \int s\chi_E. 
	\end{equation*}
\end{enumerate}
\end{defn}

\begin{rmk}%88
	Of course we think of this as ``the signed area under the curve'' as usual. 
	One irritating fine point we need to deal with is to check that if we express 
	the simple function as a linear combination of characteristic functions in a 
	different way, then $\int s$ is also given by the corresponding sum. 
\end{rmk}

\begin{pblm}\label{p:simplefuncint}%89 
	Suppose $s$ is a simple function that vanishes outside a set of finite 
	measure and $s = \sum\limits_{j=1}^nc_j\chi_{A_j}$ where the sets $\{A_j\}_{j=1}^n$ 
	are pairwise disjoint (but several different $A_j's$ may carry the same value $c_j$). 
	Then $\int s = \sum\limits_{j=1}^n c_j m(A_j).$ 

	\noindent (For any $c$ in the range of $s$, the canonical representation of $s$ 
	includes $E_c = \cup\{A_j: c_j = c\}$. We know $m(E_c) = \sum\limits_{c_j = c} m(A_j).$)
\begin{proof}
	For each $c_{j} = s(x)$ for some $x$ in the set of finite measure, let 
	$E_{j} = \cup\{ A_{i}: c_{i} = c_{j} \}$. 
	Therefore, there is a canonical representation of $s(x)$: 
	\begin{equation*}
		s = \sum\limits_{j=1}^{m} c_{j}\chi_{E_j}
	\end{equation*}	
	Where $E_{j} \cap E_{k} = \emptyset$ if $j \neq k$ 
	(as they are unions of pairwise disjoint sets), 
	and each $c_{i}$ is distinct.  Then applying the definition of the 
	integral of a simple function,
	\begin{equation*}
		\int s = \sum_{i=1}^{m} c_{i}m(E_{i}) 		
	\end{equation*}
	For any individual $E_{i}$, the measure of $E_{i}$ is the sum of the measures 
	of the $A_{j}$ which make up the union of $E_{i}$.  This is possible because all of 
	the $A_{j}$'s are pairwise disjoint, so countable subadditivity can be applied here. 
	Therefore 
	\begin{equation*}
		c_{i}m[E_{i}] = c_{i}\sum_{k=1}^{l}m[A_{k}]	.
	\end{equation*}
	This turns ( using Definition \ref{d:integral} ) into:
	\begin{equation*}
		\int s = \sum_{j=1}^{n}c_{j}m[A_{j}]
	\end{equation*}
\end{proof}
\end{pblm}

\pagebreak

\begin{pblm}\label{p:simplefuncintislinear}%90
	Suppose that $r$ and $s$ are simple functions that vanish outside a set of 
	finite measure and that $a$ and $b$ are real numbers. 
	\begin{enumerate}
		\item $\int (ar + bs) = a\int r + b \int s.$
		\item If $r \ge s$, then $\int r \ge \int s.$ 
	\end{enumerate}
	{\scriptsize{(If $s = \sum\limits_{j=1}^n c_j\chi_{E_j}$, $r = \sum\limits_{j=1}^m d_j\chi_{F_j}$ are 
	the canonical representations, consider the representations using all possible sets 
	$E_j \cap F_k$. For $(2)$, consider $r - s$.)}}
\begin{proof}~
	\begin{enumerate}
	\item Let $s = \sum\limits_{j=1}^nc_j\chi_{E_j},~ r = \sum\limits_{j=1}^md_j\chi_{F_j}$ be the canonical 
	representations for $r$ and $s$. Then let $\{G_k\}_{k=1}^{m\times n}$ be given by all of the 
	intersections  $E_j \cap F_k$ (where some intersections can be empty). 
	Then we can re-write $s$ and $r$ as 
	\begin{equation*}
		s = \sum\limits_{k=1}^{nm}c_j\chi_{G_k}~~~ 
		r = \sum\limits_{k=1}^{nm}d_j\chi_{G_k}~~~ 
	\end{equation*}
	Where, as the $G_k$ are each a subset of one and only one $E_j$, and one $F_j$, so the $c_j$ and $d_j$ 
	are not necessarily unique, but are the coefficients associated to the supersets $E_j$, $F_j$. 
	respectively. Then 
	\begin{equation*}
	\begin{array}{rcl}
		\int(ar + bs) &=& \int \left(a\sum\limits_{k=1}^{mn} d_k\chi_{G_k} + 
					   b\sum\limits_{k=1}^{mn} c_k\chi_{G_k}\right)\\
			&=& \sum\limits_{k=1}^{mn} (ad_k + bc_k) m(G_k)\\
			&=& a\sum\limits_{k=1}^{mn} d_k m(G_k) + b\sum\limits_{k=1}^{mn} c_k m(G_k)\\
			&=& a\int r + b \int s
	\end{array}
	\end{equation*}
	\item If $r \ge s$, then consider $\int (r - s)$. From part (1), we know that this 
	is equal to $\int r - \int s$, and since $r \ge s$, then $r - s \ge 0$, and so 
	we can write $r - s$ as a sum of non-negative values, and so 
	$\int (r - s) = \int r  - \int s \ge 0$, and so $\int r \ge \int s$. 
	\end{enumerate}
\end{proof}
\end{pblm}

\begin{rmk}%91
	By the preceding problem and induction, if $\{A_j\}_{j=1}^n$ is any collection of 
	sets of finite measure, not necessarily disjoint, and if 
	$s = \sum\limits_{j=1}^n c_j\chi_{A_j}$, then $\int s = \sum\limits_{j=1}^n c_j m(A_j)$ 
	so that the restriction above to pairwise disjoint sets is unnecessary. 
\end{rmk}

\begin{pblm}\label{p:infinteqsupint}%92
	Let $f$ be a bounded function defined on a measurable set $E$ with $m(E)$ finite. Then 
	\begin{equation*}
		\inf\limits_{s\ge f}\int s = \sup\limits_{r \le f}\int r
	\end{equation*}
	if and only if $f$ is measurable. \\
	{\scriptsize{(If $|f| \le M$ is measurable, partitions the range into $2n$ equal portions and use 
	this to define simple functions $r$ and $s$ with $r \le f \le s$. For the other 
	direction, let $\{r_n\}$, $\{s_n\}$ be sequences of simple functions whose integrals 
	approach the sup and inf. Then $\sup r_n\le f\le\inf s_n$, $\sup r_n$ and $\inf s_n$ 
	are measurable, and you can show $m\{x: \sup r_n \neq\inf s_n\} = 0$. )}}
\begin{proof}
	$(\Rightarrow): $ 
	Note first that if $m(E) = 0$, then $f$ is measurable, as the set 
	$\{x \in E: f(x) < \alpha\}$ has measure 0 for any $\alpha$ and so is measurable. 

	Let $\{r_n\}$, $\{s_n\}$ be sequences of step functions such that 
	\begin{equation*}
		\lim\limits_{n\to\infty} \int s_n = \inf ~~~ \& ~~~ 
		\lim\limits_{n\to\infty} \int r_n = \sup 
	\end{equation*}
	Now, since $r_n \le f \le s_m$ for all $n$ and $m$, then 
	\begin{equation*}
		\sup\limits_{n}r_n \le f \le \inf\limits_{n}s_n
	\end{equation*}
	and for all $\epsilon > 0$, there exist integers $N$ and $M$ such that for all 
	$n \ge N$ and $m \ge M$, 
	\begin{equation*}
		\left|\int r_n - \int f\right| < \frac{\epsilon}{2}m(E) ~~~ \text{ and } ~~~ 
		\left|\int s_m - \int f\right| < \frac{\epsilon}{2}m(E)
	\end{equation*}
	So 
	\begin{equation*}
		\left|\int s_p - \int r_p\right| < \epsilon \cdot m(E) ~~\forall~ p \ge max(M,N)
	\end{equation*}
	and by \mPref{p:simplefuncintislinear}, this implies that 
	\begin{equation*}
		\left|\int (s_p - r_p)\right| < \epsilon \cdot m(E)
	\end{equation*}
	and since $s_p \ge r_p$ for all $p$, this is the same as 
	\begin{equation*}
		\int (s_p - r_p) < \epsilon \cdot m(E)
	\end{equation*}
	Now, define $\tau - \epsilon ~\chi_E$. Then $\tau$ is a simple function with a 
	value of $\epsilon$ everywhere on $E$, and $0$ elsewhere. So, by 
	 \mPref{p:simplefuncintislinear}(2), 
	\begin{equation*}
		\int (s_p - r_p) < \int \tau =  \epsilon \cdot m(E) ~ \implies ~ (s_p - r_p) < \tau 
	\end{equation*}
	Therefore $s_p - r_p < \epsilon$ for all $x \in E$, and so, by \mPref{p:ext_real_val_seq}, $f$ is measurable. 

	\noindent $(\Leftarrow): $ let $f$ be measurable and bounded on $E \in \M$ where 
	$m(E)$ is finite, and let $|f| \le M$ for all $x \in E$. Then partition 
	$[-M, M]$ into $2n$ equal intervals for any $n$, so that 
	\begin{equation*}
		E_j = \left[\left.-M + \frac{M}{n}(j-1), -M + \frac{M}{n}j\right.\right)
	\end{equation*}
	From \mPref{p:contalmostsimple}, define $r_n, s_n$ as 
	\begin{equation*}
	\begin{array}{c}
		s_n = \sum\limits_{j=1}^{2n} (-M + \frac{M}{n}j) \chi_{E_j}(f(x)) \\
		r_n = \sum\limits_{j=1}^{2n} (-M + \frac{M}{n}(j-1)) \chi_{E_j}(f(x)) \\
	\end{array}
	\end{equation*}
	Then 
	\begin{equation*}
		\int(s_n-r_n) = \sum\limits_{j=1}^{2n}\left[\left(-M+\frac{M}{n}(j-1)\right)\frac{M}{n} + 
				\left(-M + \frac{M}{n}j\right)\frac{M}{n}\right] 
				= \frac{M^2}{n}
	\end{equation*}
	Now, since $M$ is fixed, as $n$ increases we have 
	\begin{equation*}
		\frac{2M^2}{n} \ge \int s_n - \int r_n \ge \inf\limits{s\ge f}\int s - \sup\limits_{r\le f}\int r \ge 0
	\end{equation*}
	and so $\inf = \sup$. 
\end{proof}
\end{pblm}

\begin{rmk} %93
	The moral of the preceding problem is that, unlike the situation with the Riemann 
	integral, we do not need to consider both upper integrals and lower integrals because 
	we already have the correct class of functions, the measurable functions. Thus we can 
	say, forget about lower integrals and just approximate from above. (This is analagous 
	to not needing inner measure to help define Lebesgue measure.)
\end{rmk}

\begin{defn}\label{d:lebesgueintegral}%94
	If $f$ is a bounded measurable function defined on a measurable set $E$ with $m(E)$ 
	finite, then the (Lebesgue) integral of $f$ is 
	\begin{equation*}
		\int f = \inf\limits_{s \ge f}\int s
	\end{equation*}
	where the inf is over the class of all simple functions $s$ with $s \ge f$. We sometimes 
	write $\int_E f$. If $E = [a, b]$, we write $\int_a^b f$.\\ 
	If $A$ is a measurable subset of $E$, we define $\int_A f = \int f\chi_A$. 
\end{defn}

\pagebreak
\begin{pblm}%95
	Let $f$ be a bounded function defined on $[a,b]$. If $f$ is Riemann integrable 
	on $[a,b]$, then $f$ is measurable and the Riemann integral $R\int_a^b f =\int_a^b f$. \\
	(A step function is a simple function.)
\begin{proof}
	If $f$ is Riemann Integrable on $[a,b]$, then 
	\begin{equation*}
		R\int_a^b f = \sup\left\{\int_a^b r\right\} = \inf\left\{\int_a^b s \right\} = \int_a^b f
	\end{equation*}
\end{proof}
\end{pblm}

\begin{pblm}\label{p:lebesguelotsofuseful}%96
	The Lebesgue integral has the following properties for bounded measurable functions defined 
	on a set $E$ of finite measure. 
	\begin{enumerate}
	\item $\int_E (af + bg) = a\int_E f + b\int_E g$, 
	\item If $f \le g$ a.e., then $\int_E f \le \int_E g$, 
	\item If $f = g$ a.e., then $\int_E f = \int_E g$, 
	\item If $\alpha \le f(x) \le \beta$ for almost all $x \in E$, then 
		$\alpha ~m(E) \le \int_E f \le \beta~ m(E)$, 
	\item If $A$ and $B$ are disjoint measurable subsets of $E$, then 
		$\int_{A\cup B} f = \int_A f + \int_B$. 
	\end{enumerate}
	{\scriptsize{(For (1) show $\int af = a\int f$ (easy) and $\int(f+g) = \int f + \int g$ (Requires some 
	manipulation of infs. You may also find \mPref{p:infinteqsupint} useful.) For (2) you can 
	show $f \le 0$ a.e. implies $\int_E f \le 0.$ (3), (4), and (5) follow from previous parts.)}}
\begin{proof}
~
\begin{enumerate}
	\item If $\alpha > 0$ and $f$ is integrable, then 
		\begin{equation*}
			\int \alpha f = \inf\limits_{s\ge\alpha f} \int s = 
			\underbrace{\alpha \inf\limits_{s/\alpha \ge f}\int \frac{s}{\alpha} = 
				    \alpha \inf\limits_{s\ge f} \int s}_
					{\text{since }\inf\text{ is over all simple functions }s} = 
			\alpha \int f
		\end{equation*}
		If on the other hand, $\alpha < 0$, then (using \ref{p:infinteqsupint}), 
		\begin{equation*}
			\int \alpha f = \inf\limits_{s\ge\alpha f} \int s = 
			\alpha \sup\limits_{r/\alpha\le f}\int \frac{r}{\alpha} = 
			\alpha \int f
		\end{equation*}
		note that $\alpha = 0$ satisfies $\int \alpha f = \alpha \int f$ trivially. 

		Now, let $f, g$ both be integrable functions. For all $r \ge f$ and $s \ge g$, 
		\begin{equation*}
		\begin{array}{rcl}
			\int (f + g) &\le& \int (r + s) = \int r + \int s\\
			\int (f + g) &\le&\inf\limits_{r\ge f, g \ge s} \int (r + s) = 
				\inf\limits_{r\ge f}\int r + \inf\limits_{s\ge g}\int s = \int f + \int g. 
		\end{array}
		\end{equation*}
		On the other hand, for all simple functions $r \ge f$ and $s \ge g$,  by \mPref{p:infinteqsupint}, 
		\begin{equation*}
		\begin{array}{rcl}
			\int f + \int g & \le & \int r + \int s = \int (r + s)\\
			\int f + \int g & \le & \inf\limits_{r\ge f, s\ge g}\int (r + s) 
			\le \inf\limits_{r+s\ge f+g}\int (r + s)= \int (f + g). 
		\end{array}
		\end{equation*}

	\item if $f \le 0$ a.e. then $\int f = \inf\limits_{s\ge f}\int s \le \int 0$ since the simple function 
		$0 \ge f$ a.e. Therefore if $f \le g$ a.e., then $(f - g) \le 0$ a.e., and so 
		$\int( f - g) = \int f - \int g  \le 0$, and so $\int f \le \int g$. 

	\item If $f = g$ a.e., then $0 = \int_E (f - g) = \int_E f - \int_E g$, and so $\int_E f = \int_E g$. 

	\item if $\alpha \le f(x) \le \beta$ for almost all $x \in E$, then by part 2, 
		\begin{equation*}\int_E \alpha \le \int_E f \le \int_E \beta\end{equation*} which is 
		\begin{equation*}\alpha \, m(E) \le \int_E f \le \beta \, m(E). \end{equation*} 

	\item if $A$ and $B$ are disjoint measurable subsets of $E$, then $f = f \chi_A + f\chi_B$ over 
		the set $A \cup B$, and so 
		\begin{equation*}
			\int_{A\cup B} f = 
			\int_{A\cup B} (f\chi_A + f\chi_B) = 
			\int_{A\cup B} f\chi_A + \int_{A \cup B} f\chi_B = 
			\int_A f + \int_B f
		\end{equation*}
\end{enumerate}
\end{proof}
\end{pblm}

\begin{rmk}%97
	It follows from part (3) and the previous problem about measurability of functions equal 
	almost everywhere that changing the values of a function in a bounded way on a countable 
	subset of $E$ does not affect $\int_E f$. Thus the Dirichlet example of a non-Riemann 
	integrable function (the characteristic function of the set $\Q_0$ of rationals in $[0,1]$) 
	is not a problem here. We have $\int_0^1 \chi_{\Q_0} = 0$. 
\end{rmk}

\begin{pblm}\label{p:intofsequenceeqlimint}%98
	Let $\{f_n\}_{n=1}^\infty$ be a sequence of measurable functions on a measurable set $E$ of 
	finite measure such that for some $M > 0$, $|f_n|\le M$ a.e. for each $n$. If 
	$\lim\limits_{n\to\infty}f_n(x)$ exists a.e. on $E$, then the function defined by 
	$\lim\limits_{n\to\infty}f_n(x)$ is equal almost everywhere on $E$ to a bounded measurable 
	function $f$ and 
	\begin{equation*}
		\int_E f = \lim\limits_{n\to\infty}\int_E f_n.
	\end{equation*}
	(Use \mPref{p:approxfunwithsteporsimple} or Egoroff's Thm \mPref{p:egoroff}). 
\pagebreak
\begin{proof}
	If $\lim\limits_{n\to\infty}f_n$ exists a.e. on $E$, then $f = \lim\limits_{n\to\infty} f_n$ 
	is measurable by \mPref{p:ext_real_val_seq}, and $|f| \le M$ a.e. 

	Then by \mPref{p:egoroff}, for any $\epsilon > 0$, there is $A \subset E$ with $m(A) < \epsilon/4M$ and 
	$f_n \rightarrow f$ uniformly on $E \setminus A$. So 
	\begin{equation*}
		\int_E (f - f_n) \le \int_E |f - f_n| = \int_{E \setminus A} |f-f_n| + \int_A|f-f_n|
	\end{equation*}
	and since $f_n\rightarrow f$ uniformly on $E \setminus A$, there is $N \in \Z$ such that 
	for all $n \ge N$, 
	\begin{equation*}
		\int_E (f - f_n) < \int_E|f-f_n|\le \int_{E \setminus A} \epsilon/2m(E\setminus A) + \int_A|f-f_n|
	\end{equation*}
	and so 
	\begin{equation*}
	\begin{array}{rclr}
		 \int_E(f-f_n) & < & \frac{\epsilon}{2m(E\setminus A)} \cdot m(E\setminus A) + \int_A|f-f_n| & \forall n \ge N\\
		 & < & \frac{\epsilon}{2} + \int_A 2M & \forall n \ge N\\
		 & < & \frac{\epsilon}{2} + m(A) ~ 2M & \forall n \ge N\\
		 & < & \frac{\epsilon}{2} + \frac{\epsilon}{4M} 2M & \forall n \ge N\\
		 \int_E(f-f_n) & < & \frac{\epsilon}{2} + \frac{\epsilon}{2} & \forall n \ge N\\
		 \int_Ef-\int_Ef_n & < & \epsilon  & \forall n \ge N\\
		 \int_Ef-\lim\limits_{n\to\infty}\int_Ef_n & < & \epsilon  & \\
	\Rightarrow~~~  \int_E f & = & \lim\limits_{n\to\infty}\int_Ef_n
	\end{array}
	\end{equation*}
\end{proof}
\end{pblm}

\begin{rmk}~ %99
\begin{enumerate}
	\item The preceding problem is a form of the Bounded Convergence Theorem-the first of 
	the important convergence theorems. It asks more than pointwise convergence (the 
	uniform boundedness of the functions), but much less than uniform convergence. 
	\item Now we'll extend things by dropping our conditions that the functions be bounded 
	and defined on a set of finite measure. We'll start by defining the integral for non-negative 
	functions only, and then writing an arbitrary function as a linear combination of these. The 
	idea for non-negative functions is to take the sup over all bounded functions non-zero on a set 
	of finite measure that are below the given function.
\end{enumerate}
\end{rmk}

\begin{defn}\label{d:thehundred}%100
	Let $f$ be a non-negative measurable function on a measurable set $E$. Then we define 
	\begin{equation*}
		\int_E f = \sup\limits_{h\le f}\int_E h
	\end{equation*}
	where the sup is over all bounded non-negative measurable functions $h \le f$ such that 
	$m\{x \in E: h(x) \neq 0\} < \infty$ and we interpret $\int_E h$ as 
	$\int_{\{x\in E:h(x)\neq0\}}h$. 
\end{defn}

\begin{pblm}\label{p:conditionsnonnegmeasu}%101
	If $f$ and $g$ are non-negative measurable functions defined on a measurable set $E$, then 
	\begin{enumerate}
		\item $\int_E \alpha f = \alpha \int_E f$ for any $\alpha \ge 0$, 
		\item $\int_E(f+g) = \int_E f + \int_E g$, 
		\item If $f \ge g$ a.e. then $\int_E f \ge \int_E$. In particular, if 
		$f \ge 0$ a.e. then $\int_E f \ge 0$. 
	\end{enumerate}
\begin{proof}
~
\begin{enumerate}
	\item If $\alpha = 0$, then $\int_E \alpha f = \int_E 0 = 0 = 0\int_E f = \alpha \int_E f$. 
		Now assume $\alpha > 0$. Then 
		$\int_E \alpha f = \sup\limits_{h\le \alpha f}\int h$, and since this is over all 
		bounded measurable functions $h\le f$. Then since $\alpha > 0$, taking the supremum 
		over $h \le \alpha f$ is the same as taking the supremum over $h / \alpha \le f$. 
		Thus this becomes 
		\begin{equation*}\sup\limits_{h/\alpha \le f} \int \alpha \frac{h}{\alpha} = \sup\limits_{h/\alpha\le f} \alpha \int h/\alpha = \sup\limits_{h\le f} \alpha \int h\end{equation*}
	\item $\int_E (f + g) \ge \int(h + k) = \int h = \int k$ for all $h, k \le f, g$ respectively. 
		Therefore 
		\begin{equation*}\int_E (f + g) \ge \sup\int_E h + \sup\int_E g = \int_E f + \int_E g.\end{equation*} 
		Also, $\int_E f + \int_E g \ge \int_E h + \int_E k = \int_E(h + k)$ for all $h, k \le f, g$ respectively. 
		So $\int_E f + \int_E g \ge \sup\int_E(h+k)$. 
		Thus \begin{equation*}\int_E f + \int_E g = \int_E (f + g).\end{equation*} 
	\item If $f \ge 0$ a.e., then for any non-negative $h \le f$, let $h_f = min\{f, h\}$ and $h_g = h - h_f$. 
		Thus $h_f$ is non-negative (as the minimum of two non-negative functions), and $h_g$ is also non-negative since 
		it is bounded below by  $f - f$ whenever $f \le h$ and $f - h$ whenever $h < f$. 
		Note that $h_g + h_f = h$, and $\int_E f = \sup\limits_{h\le f}\int_E h = \sup\limits_{h\le f}\int h_g+h_f$. 
		and since both $h_g$ and $h_f$ are bounded, non-negatice and measurable and vanish outside of 
		$A = \{x: h(x) \neq 0\}$ which has finite measure, then by \mPref{p:lebesguelotsofuseful}, the integral 
		is non-negative. Thus 
		\begin{equation*}
			\int_Ef = \sup\limits_{h\le f}\int_Eh_g+h_f \ge 0
		\end{equation*}
		and this, together with the additivity of part (2) gives us for any non-negative measurable $g \le f$, 
		$\int_E g \le \int_E f$. 
\end{enumerate}
\end{proof}
\end{pblm}

\pagebreak
\begin{pblm}\label{p:fatou}%102
	Let $\{f_n\}_{n=1}^\infty$ be a sequence of non-negative measurable functions on a 
	measurable set $E$ such that $f(x) = \lim\limits_{n\to\infty}f_n(x)$ exists for almost 
	all $x \in E$. Then $f$ is measurable and 
	\begin{equation*}
		\int_E f \le \lim\inf\int_E f_n.
	\end{equation*}
\begin{proof}
	For any $h \le f$ s.t. $h$ vanishes outside a set $A$ of finite measure, 
	let $h_n = min\{h, f_n\}$. %Each $h_n$ is measurable by \mPref{p:minmaxmeasurable}. 
	Then since $\lim\limits_{n\to\infty} f_n = f$ a.e.  
	for any $\epsilon > 0$, there is $N \in \Zp$ such that for all $n \ge N$, 
	$|f_n - f| < \epsilon$. Consider the $N \in \Zp$ such that 
	$|f_n - f| \le |h - f|$ whenever $n \ge N$. Then 
	since $f_n \le f$ and $h \le f$, and since $|f_n - f| \le |h - f|$, 
	\begin{equation*}
		h_n = min\{h, f_n\} = h  ~~~ \forall n \ge N
	\end{equation*}
	Thus $\lim\limits_{n\to\infty}h_n = h$ a.e.. Therefore by \mPref{p:intofsequenceeqlimint}, 
	$ \int_E h = \lim\limits_{n\to\infty}\int_E h_n $ 
	and so since for all $n$, $h_n \le f_n$, we have 
	\begin{equation*}
		\int_E h = \lim\limits_{n\to\infty}\int_E h_n \le \lim\inf\int_E f_n
	\end{equation*}
	Therefore 
	\begin{equation*}
		\int_E f = \sup\limits_{h\le f}\int_E h \le \lim\inf\int_E f_n
	\end{equation*}
\end{proof}
\end{pblm}

\marginnote{Think about $f_n(x) = \chi_{[n,n+1]}(x)$. The limit $f$ is just the zero function, 
	but each $f_n$ has integral equal to $1$.}
\begin{rmk}%103
	The preceding problem is known as Fatou's Lemma. You should be sure that you know of an 
	example where equality does not hold. 
\end{rmk}

\begin{pblm}\label{p:mct}%104
	Let $\{f_n\}_{n=1}^\infty$ be a non-decreasing sequence of non-negative measurable functions 
	on a measurable set $E$, and let $f(x) = \lim\limits_{n\to\infty}f_n(x)$ for each $x \in E$. 
	($f(x) = \infty$ is allowed.) Then 
	\begin{equation*}
		\int_E f = \lim\limits_{n\to\infty}\int_Ef_n 
	\end{equation*}
	where it is possible that both sides of the equation are $+\infty$. 
\begin{proof}
	%WTS: $\lim\limits_{n\to\infty}\int_E f_n \le \int_E f$ and $\lim\limits_{n\to\infty}\int_E f_n \ge \int_E f$. 
	We have $f_n \le f$ for all $n$, and so 
	\begin{equation*}
		\lim\limits_{n\to\infty}\int_E f_n \le \int_E f
	\end{equation*}
	On the other hand, using Fatou's Lemma, we have 
	$%\begin{equation*}
		\int_E f \le \lim\inf \int_E f_n = \lim\limits_{n\to\infty} \int_E f_n
	$ %\end{equation*}
	and so 
	\begin{equation*}
		\int_E f = \lim\limits_{n\to\infty}\int_E f_n
	\end{equation*}
\end{proof}
\end{pblm}

\begin{rmk}%105
	This is the monotone Convergence Theorem. It and Fatou's Lemma are the principal convergence 
	theorems for non-negative functions. 
\end{rmk}

\begin{pblm}\label{p:veryuseful}%106
~
	\begin{enumerate}[(a)]
	\item Let $\{f_k\}_{k=1}^\infty$ be a sequence of non-negative measurable functions on a 
	measurable set $E$. Then 
	\begin{equation*}
		\int_E\left(\sum\limits_{k=1}^\infty f_k\right) = 
		\sum\limits_{k=1}^\infty\left(\int_Ef_k\right).
	\end{equation*}
	\item Let $f$ be a non-negative measurable function on a measurable set $E$, and let 
	$\{A_n\}_{n=1}^\infty$ be a sequence of mutually disjoint measurable subsets of $E$. 
	Then 
	\begin{equation*}
		\int_{\bigcup\limits_{n=1}^\infty A_n}f = \sum\limits_{n=1}^\infty\int_{A_n}f.
	\end{equation*}
	\end{enumerate}
\begin{proof}~
	\begin{enumerate}[(a)]
	\item Define $g_n = \sum\limits_{k=1}^nf_k$. Then $\{g_n\}$ is a nondecreasing 
		sequence and so by \mPref{p:mct}, 
		\begin{equation*}
			\int_E\lim\limits_{n\to\infty}g_n = \lim\limits_{n\to\infty}\int_Eg_n
		\end{equation*}
		and now, we will show that $\int_E\sum\limits_{k=1}^nf_k = \sum\limits_{k=1}^n\int_Ef_k$ 
		by induction. 

		\noindent Base case: $\int_E (f + g) = \int_Ef + \int_Eg$ by \mPref{p:lebesguelotsofuseful}

		\noindent Induction step: Assume that the hypothesis is true for some $n-1$, and then 
		\begin{equation*}
		\begin{array}{rcl}
			\int_E\sum\limits_{k=1}^nf_k &=& \int_E\left[\sum\limits_{k=1}^{n-1}f_k + f_n\right] \\
						&=& \int_E\sum\limits_{k=1}^{n-1}f_k+\int_Ef_n
		\end{array}
		\end{equation*}
		So 
		\begin{equation*}
			\int_E\sum\limits_{k=1}^\infty f_k = \sum\limits_{k=1}^\infty\int_Ef_k
		\end{equation*}
	\item Let $A = \bigcup\limits_{n=1}^\infty A_n$. Then 
		\begin{equation*}
		\begin{array}{ccc}
			\int_Af & = \int_A\sum\limits_{n=1}^\infty f\chi_{A_n} = & \sum\limits_{n=1}^\infty\int_Af\chi_{A_n}\\
			\verteq & & \verteq \\
			\int_{\bigcup\limits_{n=1}^\infty A_n}f & = & \sum\limits_{n=1}^\infty\int_{A_n}f.
		\end{array}
		\end{equation*}
	\end{enumerate}
\end{proof}
\end{pblm}

\begin{pblm}\label{p:nonnegmeas_infty}%107
	Let $f$ and $g$ be non-negative measurable functions on a measurable set $E$ such that 
	$f(x) \ge g(x)$ on $E$. If $\int_E f < \infty$, then $\int_E g < \infty$, 
	$\int_E(f-g) < \infty$, and 
	\begin{equation*}
		\int_E(f-g) = \int_Ef - \int_Eg. 
	\end{equation*}
\begin{proof}~
	\begin{itemize}
	\item If $\int_E f < \infty$, then by \mPref{p:conditionsnonnegmeasu}, 
		\begin{equation*}
			\int_Eg \le \int_E f < \infty ~~ \Rightarrow ~~ \int_Eg < \infty
		\end{equation*}
	\item Since $f \ge g$ on $E$, then $0 \le f - g \le f$ on $E$, and so 
		\begin{equation*}
			\int_E (f - g) \le \int_E f < \infty
		\end{equation*}
	\item 
		\begin{equation*}
		\begin{array}{rcl}
			\underbrace{\int_E (f + g) + \int_Eg}_{\text{both non-negative}} = 
			\int_E\left[(f - g) + g\right] = \int_Ef & \Rightarrow & \int_Ef = 
			\int_E(f-g)+\int_Eg\\
			& \Rightarrow & \int_Ef - \int_Eg = \int_E(f - g)
		\end{array}
		\end{equation*}
	\end{itemize}
\end{proof}
\end{pblm}

\begin{pblm}%108
	Let $f$ be a non-negative measurable function on a mesaurable set $E$ such that 
	$\int_E f < \infty$. Then for each $\epsilon > 0$ there is $\delta > 0$ so that 
	for every $A \subset E$ with $m(A) < \delta$, it is the case that $\int_Af<\epsilon$. 
\begin{proof}
	Assume not, then for some $\epsilon > 0$, $\forall \delta > 0$ there are 
	$A \subset E$ with $m(A) \le \delta$ such that 
	\begin{equation*}
		\int_A f \ge \epsilon.
	\end{equation*}
	Let this $\epsilon > 0$, then there is a sequence $\{A_n\}$ with 
	$m(A_n) < \frac{1}{2^n}$ and $\int_{A_n} f \ge \epsilon$. 

	Let $B_n =\bigcup\limits_{k=n}^\infty A_k$, and define $f_n = f - f\chi_{B_n}$. 

	Then $f_n \rightarrow f$ (obv), and so 
	then $f_n$ is a non-decreasing sequence of non-negative ftns (that are measurable), 

	Now, 
	\begin{equation*}
	\begin{array}{rcl}
		\int_E (f - f_n) & = & \int_E f - \int_E f_n \\
		& = & \int_E f - \int_E (f - f\chi_{B_n}) \\
		& = & \int_E f - \int_E f + \int_E f\chi_{B_n} = \int_E f\chi_{B_n} = \int_{B_n} f
	\end{array}
	\end{equation*}
	and so 
	\begin{equation*}
		\lim\limits_{n\to\infty} \int_E (f - f_n) = \lim\limits_{n\to\infty} \int_{B_n} f \ge \epsilon > 0
	\end{equation*}
	Which is a contradiction, as $f_n \rightarrow f$. 
\end{proof}
\end{pblm}

\begin{rmk}%109
	This looks like a sort of continuity condition. As the proof demonstrates, it 
	really says that an integrable function cannot have ``delta function'' 
	parts-places of size zero but positive area (or positive mass if you prefer to 
	think of $f$ as representing density). 
\end{rmk}

\begin{rmk}%110
	Let $f$ be an extended real-valued function. The positive and negative parts of 
	$f$, $f_+ = max\{f, 0\}$ and $f_- = max\{-f, 0\}$ can be defined as extended 
	real-valued functions just as in \mDref{d:pos_neg_fun} with the properties already 
	developed. Notice that for nay $x$, at most one of $f_+$ and $f_-$ is different 
	from 0, and that, as before, $f = f_+ - f_-$ and $|f| = f_+ + f_-$. 
\end{rmk}

\begin{defn}%111
	~
	\begin{enumerate}
	\item A non-negative extended real-valued measurable function $f$ defined on a 
	measurable function $f$ defined on a measurable set $E$ is \textbf{integrable over} 
	$E$ if $\int_Ef<\infty$. 
	\item An extended real-valued measurable function $f$ defined on a measurable 
	set $E$ is \textbf{integrable over} $E$ if both its positive part and its 
	negative part, $f_+$ and $f_-$, are integrable over $E$. In that case, we define 
	\begin{equation*}
		\int_E f = \int_Ef_+-\int_Ef_-.
	\end{equation*}
	We denote the set of all functions integrable over $E$ by $L^1(E)$. 
	\end{enumerate}
\end{defn}

\begin{rmk}%112
~
\begin{enumerate}
	\item An integrable extended real-valued function $f$ must have finite values 
	almost everywhere. 
	\item A measurable function $f$ on $E$ is integrable over $E$ if and only if 
	$|f| = f_++f_-$ is integrable over $E$. Thus $L^1(E)$ is usually defined as the 
	set of all measurable functions on $E$ such that $\int_E|f|<\infty$. 
\end{enumerate}
\end{rmk}

\begin{pblm}\label{p:conditionsmeasu}%113
	If $f$ and $g$ are integrable on $E$, and if $\alpha$, $\beta$ are real 
	numbers, then 
	\begin{enumerate}[(a)]
	\item $\int_E(\alpha f + \beta g) = \alpha \int_Ef + \beta\int_Eg$. 
	\item if $f \le g$, then $\int_Ef \le \int_Eg$, 
	\item if $A$ and $B$ are disjoint measurable subsets of $E$, then 
	$\int_{A\cup B}f = \int_Af + \int_Bf$. 
	\end{enumerate}
\begin{proof} ~
	\begin{enumerate}[(a)]
	\item if $|f|=f_++f_-$, then, if $\alpha >= 0$,
	\begin{equation*}
		\int_E\alpha f = \int_E\alpha(f_+-f_-) = \int_E[\alpha f_+ - \alpha f_-] 
	\end{equation*}
	and by \mPref{p:nonnegmeas_infty}, this is 
	\begin{equation*}
	\begin{array}{rcl}
		\int_E\alpha f = \int_E[\alpha f_+ - \alpha f_-] & = & \int_E\alpha f_+ - \int_E \alpha f_- \\
			& = & \alpha\int_E f_+ - \alpha\int_E f_- \\ 
			& = & \alpha\left[\int_Ef_+ - \int_Ef_-\right]\\
			& = & \alpha\int_Ef. 
	\end{array}
	\end{equation*}
	If on the other hand, $\alpha < 0$, then 
	\begin{equation*}
	\begin{array}{rcl}
		\int_E \alpha f = \int_E (-\alpha) (-f) & = & (-\alpha) \int_E (-f) \\
			& = & -\alpha \left[\int_E f_- - \int_E f_+\right]\\
			& = & -\alpha \int_E f_- + \alpha \int_E f_+\\
			& = & \alpha \int_E f_+ - \alpha \int_E f_-\\
			& = & \alpha \int_E f. 
	\end{array}
	\end{equation*}
	Now, to show that $\int_E(f+g) = \int_Ef+\int_Eg$, 
	\begin{equation*}
	\begin{array}{rcl}
		\int_E(f+g) = \int_E[f_++g_+-f_--g_-] &=& \int_Ef_++\int_Eg_+-\int_Ef_--\int_Eg_- \\ 	
			& = & \int_Ef_+-\int_Ef_-+\int_Eg_+-\int_Eg_- \\ 	
			& = & \int_E(f_+-f_-) + \int_E(g_+-g_-) \\
			& = & \int_Ef + \int_Eg. 
	\end{array}
	\end{equation*}
	\item If $g \ge f$, then $g - f \ge 0$, which (using the first part of this problem) implies that 
		\begin{equation*}
			0 \le \int_E(g - f) = \int_Eg - \int_E f ~~\Rightarrow~~ \int_Eg \ge \int_Ef 
		\end{equation*}
	\item 
	\begin{equation*}
		\int_{A\cup B} f = \int_{A\cup B} \left(f \chi_A + f\chi_B\right) = 
		\int_{A\cup B} f\chi_A + \int_{A\cup B} f\chi_B = \int_A f + \int_B f
	\end{equation*}
	\end{enumerate}
\end{proof}
\end{pblm}

\pagebreak
\begin{pblm}\label{p:lebesguedominatedconvergence}%114
	Let $g$ be a non-negative integrable function on $E$, and let $\{f_n\}_{n=1}^\infty$ 
	be a sequence of measurable functions such that $|f_n| \le g$ a.e. on $E$ for each 
	$n$. If $f_n(x) \rightarrow f(x)$ for almost all of $x \in E$, then 
	\begin{equation*}
		\lim\limits_{n\to\infty}\int_Ef_n = \int_Ef. 
	\end{equation*}
\begin{proof}
	If $|f_n| \le g$ a.e. for all $n$, then 
	$\{g - f_n\}_{n=1}^\infty$ and $\{g + f_n\}_{n=1}^\infty$ are both non-negative a.e. 
	and so by Fatou's Lemma (\mPref{p:fatou}), 
	\begin{equation*}
	\begin{array}{rcl}
		\int_E(g+f) & \le & \lim\inf\int_E(g+f_n) \\ 
		\int_Eg + \int_E f & \le & \lim\limits_{n\to\infty}\int_E(g+f_n)\\
		\int_Eg + \int_Ef & \le & \int_Eg + \lim\limits_{n\to\infty}\int_Ef_n\\
		\int_Ef & \le & \lim\limits_{n\to\infty}\int_Ef_n
	\end{array}
	\end{equation*}
	and similarly, 
	\begin{equation*}
	\begin{array}{rcl}
		\int_E(g-f) & \le & \lim\inf\int_E(g-f_n) \\ 
		\int_Eg - \int_E f & \le & \lim\limits_{n\to\infty}\int_E(g-f_n)\\
		\int_Eg - \int_Ef & \le & \int_Eg - \lim\limits_{n\to\infty}\int_Ef_n\\
		-\int_Ef & \le & -\lim\limits_{n\to\infty}\int_Ef_n\\
		\int_Ef & \ge & \lim\limits_{n\to\infty}\int_Ef_n
	\end{array}
	\end{equation*}
\end{proof}
\end{pblm}

\begin{rmk}%115
	The preceeding result is the Lebesgue Dominated Convergence Theorem. It is perhaps the 
	most often used of the convergence theorems. Note that our first convergence theorem 
	\mPref{p:intofsequenceeqlimint}, is a special case. 
\end{rmk}

\begin{rmk}%116
	Note that $f$ being Lebesgue integrable over a set $E$ is not the same as $f$ having an 
	improper Riemann integral over $E$. For instance, $\int_0^\infty \frac{\sin x}{x}dx$ 
	exists, as you may see by observing that if we set $a_n = \int_{(n-1)\pi}^{n\pi}\frac{\sin x}{x}dx$ 
	then $\int_0^\infty\frac{\sin x}{x}dx = \sum\limits_{n=1}^\infty a_n$ converges by the 
	Alternating Series Test. However the integral of the positive part of $\frac{\sin x}{x}$ is 
	infinite (and the integral of the negative part also) since $|a_n|\approx \frac{k}{n}$. 
	In other words, $\sum\limits_{n=1}^\infty a_n$ converges conditionally, but not absolutely. 
	$f$ being Lebesgue integrable corresponds to absolute convergence-convergence via cancellation 
	is not allowed. This makes working with Lebesgue integrable functions much simpler, at the 
	expense of making the class of integrable functions somewhat smaller. 
\end{rmk}


